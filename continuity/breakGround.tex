\documentclass{ximera}

\input{../preamble.tex}

\outcome{Understand the necessity of continuity for the Intermediate Value Theorem.}
\outcome{Determine if the Intermediate Value Theorem applies.}

\title[Break-Ground:]{Gabby and Dustin like food}

\begin{document}
\begin{abstract}
Two young mathematicians discuss the eating habits of their cats.
\end{abstract}
\maketitle

Check out this dialogue between two calculus students (based on a true
story):

\begin{dialogue}
\item[Devyn] Yo Riley, I was watching my two cats
  \textit{Gabby} and \textit{Dustin} eat their dry cat food last night.
\item[Riley] Cats love food!  It's so weird that they swallow the pieces whole!
\item[Devyn] I know! I noticed something else kinda funny though:
  Both Gabby and Dustin start and finish eating at the same times; and
  while I gave Gabby a little more food than Dustin, less food was left
  in Gabby's bowl when they stopped eating.

  I wonder, is there is a point in time when Gabby and Dustin have the
  exact same amount of \textbf{dry cat food} in their bowls?
\item[Riley] Hmmmmm. Do Gabby and Dustin both start and finish
  drinking their water at the same times?  And does Gabby start with a
  little more water than Dustin, and finish with less water left than
  Dustin?
\item[Devyn] Yes!
\item[Riley] Interesting. I wonder, is there is a point in
  time when Gabby and Dustin have the exact same amount of \textbf{water}
  in their bowls?
\end{dialogue}

\begin{problem}
  Is there a time when Gabby and Dustin have the same amount of dry cat
  food in their bowls assuming:
  \begin{itemize}
  \item They start and finish eating at the same times.
  \item Gabby starts with more food than Dustin, and leaves less food uneaten than Dustin. 
  \end{itemize}
  \begin{hint}
  	You might want to try drawing a graph of this situation.
  \end{hint}
  \begin{multipleChoice}
    \choice{yes}
    \choice{no}
    \choice[correct]{There is no way to tell.}
  \end{multipleChoice}
\end{problem}

\begin{problem}
  Is there a time when Gabby and Dustin have the same amount of water in
  their bowls assuming:
  \begin{itemize}
  \item They start and finish drinking at the same times.
  \item Gabby starts with more water than Dustin, and leaves less water
    left in her bowl than Dustin.
  \end{itemize}
    \begin{hint}
  	You might want to try drawing a graph of this situation.
  \end{hint}
    \begin{multipleChoice}
    \choice[correct]{yes}
    \choice{no}
    \choice{There is no way to tell.}
  \end{multipleChoice}
\end{problem}



\begin{problem}
  Within the context of the two problems above, what is the difference
  between ``dry cat food'' and ``water?''
  \begin{freeResponse}
    If we write the amount of dry cat food as a function of time, this function
    is not continuous.  The reason it isn't continuous is that the dry cat food
    is a collection of individual kibbles, which are eaten whole.
    
    On the other hand, if we write the amount of water as a function of time, 
    this function is continuous.
  \end{freeResponse}
\end{problem}

\input{../leveledQuestions.tex}


\end{document}
