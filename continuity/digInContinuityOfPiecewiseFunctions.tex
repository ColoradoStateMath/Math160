\documentclass{ximera}

\input{../preamble.tex}

\outcome{Identify where a function is, and is not, continuous.}
\outcome{Understand the connection between continuity of a function and
  the value of a limit.}
\outcome{Make a piece-wise function continuous.}

\title[Dig-In:]{Continuity of piecewise functions}

\begin{document}
\begin{abstract}
Here we use limits to ensure piecewise functions are continuous.
\end{abstract}
\maketitle

In this section we will work a couple of examples involving limits,
continuity and piecewise functions.

\begin{example}
Consider the following piecewise defined function
\[
f(x) = 
\begin{cases}
  \frac{x}{x-1} &\text{if $x<0$,}\\
  \cos(-x) + C &\text{if $x\ge 0$}.
\end{cases}
\]
Find $C$ so that $f$ is continuous at $x=0$.
\begin{explanation}
  To find $C$ such that $f$ is continuous at $x=0$, we need to find
  $c$ such that
  \[
  \lim_{x\to 0^-} f(x) = \lim_{x\to 0^+}f(x) = f(\answer[given]{0}).
  \]
  In this case
  \begin{align*}
    \lim_{x\to 0^-} f(x) &= \lim_{x\to 0^-}\answer[given]{\frac{x}{x-1}}\\
    &= \answer[given]{\frac{0}{-1}}\\
    &=\answer[given]{0}.
  \end{align*}
  On there other hand
  \begin{align*}
    \lim_{x\to 0^+} f(x) &= \lim_{x\to 0^+}\left(\answer[given]{\cos(-x)+C}\right)\\
    &= \answer[given]{\cos(0) + C}\\
    &= \answer[given]{1+ C}
  \end{align*}
  Hence for our function to be continuous, we need
  \[
  1 + C = 0\qquad\text{so}\qquad C = \answer[given]{-1}.
  \]
  Now, $\displaystyle\lim_{x\to 0} f(x) = f(\answer[given]{0})$, and so $f$ is continuous.
\end{explanation}
\end{example}


Consider the next, more challenging example.

\begin{example}
Consider the following piecewise defined function
\[
f(x) = 
\begin{cases}
  x+4 &\text{if $x<1$,}\\
  ax^2+bx+2 &\text{if $1\le x< 3$,}\\
  6x+a-b &\text{if $x\ge 3$.}
\end{cases}
\]
Find $a$ and $b$ so that $f$ is continuous at both $x=1$ and $x=3$.
\begin{explanation}
This problem is more challenging because we have more
unknowns. However, be brave intrepid mathematician.  To find $a$ and
$b$ that make $f$ is continuous at $x=1$, we need to find $a$ and $b$
such that
\[
\lim_{x\to 1^-} f(x) = \lim_{x\to 1^+}f(x)=f(\answer[given]{1}).
\]
Looking at the limit from the left, we have
\begin{align*}
  \lim_{x\to 1^-} f(x) &= \lim_{x\to 1^-} \left(\answer[given]{x+4}\right) \\
  &=\answer[given]{5}.
\end{align*}
Looking at the limit from the right, we have
\begin{align*}
  \lim_{x\to 1^+} f(x) &= \lim_{x\to 1^+} \left(\answer[given]{ax^2+bx+2}\right) \\
  &= \answer[given]{a+b+2}.
\end{align*}
Hence for this function to be continuous at $x=1$, we must have that
\begin{align*}
  5 &= a+b+2\\
  3 &= \answer[given]{a+b}.
\end{align*}
Hmmmm. More work needs to be done.

To find $a$ and $b$ that make $f$ is continuous at $x=3$, we need to
find $a$ and $b$ such that
\[
\lim_{x\to 3^-} f(x) =\lim_{x\to 3^+}f(x) =f(\answer[given]{3}).
\]
Looking at the limit from the left, we have
\begin{align*}
  \lim_{x\to 3^-} f(x) &= \lim_{x\to 3^-} \left(\answer[given]{ax^2+bx+2}\right) \\
  &=\answer[given]{a\cdot 9 + b\cdot 3 + 2}.
\end{align*}
Looking at the limit from the right, we have
\begin{align*}
  \lim_{x\to 3^+} f(x) &= \lim_{x\to 3^+} \left(\answer[given]{6x+a-b}\right) \\
  &= \answer[given]{18+a-b}.
\end{align*}
Hence for this function to be continuous at $x=3$, we must have that
\begin{align*}
  a\cdot 9 + b\cdot 3 + 2 &= 18+a-b\\
  a\cdot 8 + b\cdot 4 -16 &= 0\\
  a\cdot 2 + b -4 &= 0
\end{align*}
So now we have two equations and two unknowns:
\[
 3=a+b \qquad\text{and}\qquad a\cdot 2 + b -4 = 0.
 \]
 Set $b = 3-a$ and write
 \begin{align*}
   0&= a\cdot 2 + (3-a) -4 \\
   &= a -1,
 \end{align*}
 hence
 \[
 a=\answer[given]{1}\qquad\text{and so}\qquad b =\answer[given]{2}.
 \]
 Let's check, so now plugging in values for both $a$ and $b$ we find
 \[
 f(x) = 
 \begin{cases}
   x+4 &\text{if $x<1$,}\\
   \answer[given]{x^2+2x+2} &\text{if $1\le x< 3$,}\\
   \answer[given]{6x-1} &\text{if $x\ge 3$.}
\end{cases}
 \]
 Now
 \[
 \lim_{x\to 1^-} f(x) =\lim_{x\to 1^+}f(x) =f(1) =  5,
 \]
 and
 \[
 \lim_{x\to 3^-} f(x) =\lim_{x\to 3^+}f(x) =f(3) = 17.  
 \]
 So setting $a= \answer[given]{1}$ and $b=\answer[given]{2}$ makes $f$ continuous at $x=1$ and $x=3$.
 \begin{prompt}
   We can confirm our results by looking at the graph of $y=f(x)$:
   \[
   \graph{y=\\\{x<1: x+4\\\},y=\{1 \leq x<3: x^2+2x+2\},y=\{x \geq 3: 6x-1\}}
   \]
 \end{prompt}
\end{explanation}
\end{example}



\end{document}
