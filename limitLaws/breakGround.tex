\documentclass{ximera}

\input{../preamble.tex}

\outcome{}


\title[Break-Ground:]{Equal or not?}

\begin{document}
\begin{abstract}
Here we see a dialogue where students discuss combining limits with
arithmetic.
\end{abstract}
\maketitle


Check out this dialogue between two calculus students (based on a true
story):

\begin{dialogue}
\item[Devyn] Riley, I've been thinking about limits.
\item[Riley] So awesome!
\item[Devyn] Think about
  \[
  \displaystyle\lim_{x\to a} \left(f(x) + g(x)\right).
  \]
  This is the number that $f(x) + g(x)$ gets nearer and nearer to, as $x$ gets nearer and nearer to $a$. 
\item[Riley] You know it!
\item[Devyn] So I think it is the same as
  \[
  \displaystyle\lim_{x\to a} f(x) + \displaystyle\lim_{x\to a}g(x).
  \]
\item[Riley] Yeah, that does make sense, since when you add two
  numbers, say
  \[
  (\text{a number near $6$}) + (\text{a number near $7$})
  \]
  you get
  \[
  (\text{a number near $13$})
  \]
\item[Riley] Right! And I think the same reasoning will work for
  multiplication! So we should be able to say
  \[
  \displaystyle\lim_{x\to a}\left(f(x) \cdot g(x)\right) = \left(\displaystyle\lim_{x\to a} f(x) \right)\cdot\left(\displaystyle\lim_{x\to a} g(x)\right).
  \]
\item[Devyn] Yes, I think that's right! But what about
  \textit{division}? Can we use similar reasoning to conclude
  \[
  \displaystyle\lim_{x\to a} \frac{f(x)}{g(x)} = \displaystyle\frac{\lim_{x\to a}
    f(x)}{\displaystyle\lim_{x\to a} g(x)}.
  \]
\end{dialogue}



\begin{problem}
  Give an argument (similar to the one above) supporting the idea that
  \[
  \displaystyle\lim_{x\to a}\left(f(x) \cdot g(x)\right) = \left(\displaystyle\lim_{x\to a} f(x) \right)\cdot\left(\displaystyle\lim_{x\to a} g(x)\right).
  \]
  \begin{freeResponse}
  \end{freeResponse}
\end{problem}


For the next problems, suppose $L$ is a number near $1$ and that $M$
is a number near $0$.


\begin{problem}
  Using the context above, 
  \[
  \frac{\text{large}}{\text{small}} = ?
  \]
  \begin{multipleChoice}
  \choice[correct]{``large''}
  \choice{``small''}
  \choice{impossible to say}
  \end{multipleChoice}
\end{problem}

\begin{problem}
  Using the context above, 
  \[
  \frac{\text{small}}{\text{small}} = ?
  \]
  \begin{multipleChoice}
  \choice{``large''}
  \choice{``small''}
  \choice[correct]{impossible to say}
  \end{multipleChoice}
\end{problem}

\input{../leveledQuestions.tex}


\end{document}
