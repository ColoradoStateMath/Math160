\documentclass{ximera}

\input{../preamble}

\outcome{To be able to use the method of substitution to solve more difficult types of integrals.}
\outcome{To be able to both correctly identify what to substitute for and to be able to successfully carry out the process to correctly solve the problem.}

\title[Dig-In:]{Working with substitution}

\begin{document}
\begin{abstract}
We explore more difficult problems involving substitution.
\end{abstract}
\maketitle


We begin by restating the substitution formula.


\begin{theorem}[Integral Substitution Formula] 
If $u$ is differentiable on the interval $[a,b]$ and $f$ is
differentiable on the interval $[u(a),u(b)]$, then
\[
\int_a^b f'(u(x)) u'(x) \d x =\int_{u(a)}^{u(b)} f'(u) \d u.
\]
\end{theorem}

We spend pretty much this entire section working out examples.


\begin{example}
Compute:
\[
\int_{5}^{8}\sqrt{x-4} \d x
\]
\begin{explanation}
  Let
  \[
  u =\answer[given]{x-4},
  \]
  computing $\d u$, we find
  \[
  \d u =\answer[given]{1}\d x
  \]
  
  Now
\begin{align*}
\int_{5}^{8}\sqrt{x-4} \d x &= \int_{u(5)}^{u(8)} \sqrt{u}\d u  \\
&= \int_{\answer[given]{1}}^{\answer[given]{4}} \answer[given]{\sqrt{u}} \d u\\
&= \eval{\answer[given]{\frac{2}{3}}u^{3/2}}_{\answer[given]{1}}^{\answer[given]{4}}\\
&= \frac{2}{3}\left(\answer[given]{4}^{3/2}-\answer[given]{1}^{3/2}\right)\\
&= \frac{2}{3}(\answer[given]{8}-\answer[given]{1})\\
&= \answer[given]{14/3}.
\end{align*}
\end{explanation}
\end{example}


The next example requires a new technique.

\begin{example}
  Compute:
\[
\int x(3+x)^{14}\d x
\]
\begin{explanation}
Here it is not apparent that the chain rule is involved. However, if
it was involved, perhaps a good guess for $u$ would be
\[
u = \answer[given]{3+x}
\]
and then we can write

\begin{align*}
    \int x(3+x)^{14}\d x &= \int(u-3)(u)^{14}\d u\\
    &= \int u^{\answer[given]{15}}-3u^{\answer[given]{14}}\d u
\end{align*}

Now we can use the power rule.

\begin{align*}
    \int x(3+x)^{14}\d x &= \answer[given]{1/(16)}u^{\answer[given]{16}}\\
    &-\answer[given]{3/(15)}u^{\answer[given]{15}}\\
    &= \frac{1}{16}(\answer[given]{3+x})^{\answer[given]{16}}\\
    &-\frac{1}{5}(\answer[given]{3+x})^{\answer[given]{15}}+C.
\end{align*}
\end{explanation}
\end{example}


\begin{example}
Compute:
\[
\int \frac{\sin(\sqrt{x})}{\sqrt{x}}\d x
\]
\begin{explanation}
What should we let $u$ be?
\[
u = \answer[given]{\sqrt{x}}
\]

So we get the derivative of $u$ is 
\[
\d u = \answer[given]{\frac{1}{2\sqrt{x}}} \Rightarrow \answer[given]{2}du \\
\]
\[
\phantom{\d u = \frac{1}{2\sqrt{x}} \Rightarrow 2 du} =\answer[given]{\frac{1}{\sqrt{x}}} \d x
\]

The integral resulting after substitution is
\begin{align*}
\int \frac{\sin(\sqrt{x})}{\sqrt{x}}\d x &= \int\answer[given]{2}\sin(\answer[given]{u})\d u\\
&= \answer[given]{-2}\cos(u)\\
&=-2\cos(\answer[given]{\sqrt{x}})+C
\end{align*}


\end{explanation}
\end{example}
\end{document}
