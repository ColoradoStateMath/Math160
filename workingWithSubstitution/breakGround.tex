\documentclass{ximera}

\input{../preamble.tex}

\outcome{Practice the mechanical process of substitution.}


\title[Break-Ground:]{Integrals are puzzles!}

\begin{document}
\begin{abstract}
Two young mathematicians discuss how tricky integrals are puzzles.
\end{abstract}
\maketitle

Check out this dialogue between two calculus students (based on a true
story):

\begin{dialogue}
\item[Devyn] Yo Riley, is it just me, or are integrals kind of fun?
\item[Riley] I always feel accomplished when I finish one.
\item[Devyn] I know! Also, even though antiderivatives are difficult,
  we can always check our work by taking the derivative.
\item[Riley] So awesome!
\item[Devyn] But something is bothering me. When we are doing substitution,
  we have to find $f$ and $u$ such that
  \[
  \int f(u(x)) \cdot u'(x) \d x = \int f(u) \d u.
  \]
  How do we choose $f$ and $u$?
\item[Riley] Well, never ever pick $u(x) = x$, this doesn't change
    anything!
\item[Devyn] And never ever pick $u(x)$ to be the entire integrand,
  this doesn't help either.
\item[Riley] Somehow we must ``see'' one function ``nested'' inside of
  another.
\item[Devyn] I'm not sure there's an easy path to doing, this, I think
  it's gonna take practice.
\end{dialogue}

%% Maybe we pick several problems and make the students identify f and
%% u based soley on u or f being given... This should help with the
%% skill of finding f and u.

In the problems that follow, we will be using the substitution formula
\[
   \int f(u(x)) \cdot u'(x) \d x = \int f(u) \d u
\]
While you may use a slightly different method to compute your
integrals, the skills developed by answering the problems below will
help you in your quest to conquer calculus.

\begin{problem}
  Consider
  \[
  \int \sin^5(3x) \cos(3x) \d x = \int f(u(x)) \cdot u'(x) \d x
  \]
  if $u(x) = 3x$, and 
  \[
  \int f(u(x)) \cdot u'(x) \d x = \int f(u) \d u.
  \]
  what is $f(u)$?
  \begin{prompt}
    \[
    f(u) = \answer{
      \frac{\sin^5(u) \cos(u)}{3}}
    \]
  \end{prompt}
  
  But we still have not yet integrated... Now we have 
  \[
    \int \frac{\sin^5(u) \cos(u)}{3} \d u
    \]
  
\end{problem}

\begin{problem}
  Consider
  \[
  \int \sin^5(3x) \cos(3x) \d x = \int f(u(x)) \cdot u'(x) \d x
  \]
  if $u(x) = \sin(3x)$, then $du=3\cos(3x)\Rightarrow \frac{1}{3}du=\cos(3x)dx$
  \[
  \int \frac{1}{3} u^5 \d u=\frac{1}{3}\cdot\answer[given]{\frac{1}{6}\sin^6(3x)}+C.
  \]


\end{problem}

\begin{problem}
	In your own words, explain why Devyn and Riley claim we should never 
	pick $u(x) = x$ or $u(x)$ to be the entire integrand.
	\begin{freeResponse}
		The goal of substitution is to make the integral easier to do.  Your choices
		for $f$ and $u$ should make things easier, not harder!
	
		Unless the derivative of $u(x)$ is $1$, choosing $u(x)$ to be the entire
		integrand means that you don't have any part of the integrand left to be
		the derivative of $u$.  Choosing $u(x) = x$ means that $u'(x) = 1$, meaning
		that you haven't simplified the integral at all.  
	\end{freeResponse}
\end{problem}

\input{../leveledQuestions.tex}


\end{document}
