\documentclass{ximera}

\input{../preamble.tex}

\outcome{Use limits to find the slope of the line tangent to a curve at a point.}
\outcome{Understand the definition of the derivative at a point.}

\title[Break-Ground:]{Slope of a curve}

\begin{document}
\begin{abstract}
Two young mathematicians discuss the novel idea of the ``slope of a curve.''
\end{abstract}
\maketitle


Check out this dialogue between two calculus students (based on a true
story):

\begin{dialogue}
\item[Devyn] Riley, do you remember ``slope?'
\item[Riley] Most definitely. ``Rise over run.''
\item[Devyn] You know it.
\item[Riley] ``Change in $y$ over change in $x$.''
\item[Devyn] That's right.  
\item[Riley] Brought to you by the letter ``$m$.''
\item[Devyn] Enough! My important question is: could we define
  ``slope'' for a curve that's not a straight line?
\item[Riley] Well, maybe if we ``zoom in'' on a curve, it would look
  like a line, and then we could call it ``the slope at that point.''
\item[Devyn] Ah! And this ``zoom in'' idea sounds like a limit!
\item[Riley] This is so awesome. We just made math!
\end{dialogue}

The concept introduced above, of the ``slope of a curve at a point,''
is in fact one of the central concepts of calculus. It will,
of course, be completely explained.  Let's explore Devyn
and Riley's ideas a little more, first.



To find the ``slope of a curve at a point,'' Devyn and Riley spoke of
``zooming in'' on a curve until it looks like a line. When you zoom in
on a \textit{smooth} curve, it will eventually look like a line. This
line is called the tangent line. \textit{It is the line tangent to} $f(x)$ \textit{at} $x=a$.

\begin{image}
\begin{tikzpicture}
  \begin{axis}[
            domain=0:6, range=0:7,
            ymin=-.2,ymax=7,
            width=6in,
            height=2.5in, %% Hard coded height! Moreover this effects the aspect ratio of the zoom--sort of BAD
            axis lines=none,
          ]   
          \addplot [draw=none, fill=textColor!10!background] plot coordinates {(.8,1.25) (5.334,3.666)} \closedcycle; %% zoom fill
          \addplot [draw=none, fill=textColor!10!background] plot coordinates {(5.334,3.666) (6.666,3.666)} \closedcycle; %% zoom fill
          \addplot [draw=none, fill=background] plot coordinates {(1.2,1.25) (6.666,3.666)} \closedcycle; %% zoom fill
          \addplot [draw=none, fill=background] plot coordinates {(.8,1.25) (1.2,1.25)} \closedcycle; %% zoom fill

          \addplot [draw=none, fill=textColor!10!background] plot coordinates {(.8,.833) (5.334,2.334)} \closedcycle; %% zoom fill
          \addplot [draw=none, fill=textColor!10!background] plot coordinates {(5.334,2.334) (6.666,2.334)} \closedcycle; %% zoom fill
          \addplot [draw=none, fill=background] plot coordinates {(.8,.833) (6.666,2.334)} \closedcycle; %% zoom fill

          \addplot[very thick,penColor, smooth,domain=(0:1.833)] {-1/(x-2)};


          \addplot[very thick,penColor, smooth,domain=(5.334:6.666)] {x-3}; %% 5 to 6.833
          
          \addplot[color=penColor,fill=penColor,only marks,mark=*] coordinates{(1,1)};  %% point to be zoomed

          \addplot [dashed] plot coordinates {(6,2.334) (6,3)}; %% zoom fill
          \addplot [dashed] plot coordinates {(5.334,3) (6,3)}; %% zoom fill
          \addplot[color=penColor,fill=penColor,only marks,mark=*] coordinates{(6,3)};  %% zoomed pt 2 

          \addplot [->,textColor] plot coordinates {(0,0) (0,6)}; %% axis
          \addplot [->,textColor] plot coordinates {(0,0) (2,0)}; %% axis
          
          \addplot [textColor!50!background] plot coordinates {(.8,.833) (.8,1.25)}; %% box around pt
          \addplot [textColor!50!background] plot coordinates {(1.2,.833) (1.2,1.25)}; %% box around pt
          \addplot [textColor!50!background] plot coordinates {(.8,1.25) (1.2,1.25)}; %% box around pt
          \addplot [textColor!50!background] plot coordinates {(.8,.833) (1.2,.833)}; %% box around pt
          
          \addplot [textColor] plot coordinates {(5.334,2.334) (5.334,3.666)}; %% zoomed box 2
          \addplot [textColor] plot coordinates {(6.666,2.334) (6.666,3.666)}; %% zoomed box 2
          \addplot [textColor] plot coordinates {(5.334,2.334) (6.666,2.334)}; %% zoomed box 2
          \addplot [textColor] plot coordinates {(5.334,3.666) (6.666,3.666)}; %% zoomed box 2

          \node at (axis cs:6,2.334) [anchor=north] {$a$};
          \node at (axis cs:6.666,2.334) [anchor=north] {$a+h$};
          \node at (axis cs:5.334,2.334) [anchor=north] {$a-h$};

          \node at (axis cs:5.334,3) [anchor=east] {$f(a)$};
          \node at (axis cs:5.334,3.666) [anchor=east] {$f(a+h)$};
          \node at (axis cs:5.334,2.334) [anchor=east] {$f(a-h)$};
          
          \node at (axis cs:2.2,0) [anchor=east] {$x$};
          \node at (axis cs:0,6.6) [anchor=north] {$y$};
        \end{axis}
\end{tikzpicture}
\end{image}



\begin{problem}
  Which of the following approximate the slope of the ``zoomed line''? Select all that are correct.
  \begin{selectAll}
    \choice{$\frac{(f(a)+h) - f(a)}{(a+h)-a}$}
    \choice[correct]{$\frac{f(a+h) - f(a)}{(a+h)-a}$}
    \choice{$\frac{(f(a)-h) - f(a)}{(a-h)-a}$}
    \choice[correct]{$\frac{f(a-h) - f(a)}{(a-h)-a}$}
    \choice{$\frac{f(a) - (f(a)+h)}{a-(a+h)}$}
    \choice[correct]{$\frac{f(a) - f(a+h)}{a-(a+h)}$}
    \choice{$\frac{f(a) - (f(a)-h)}{a-(a-h)}$}
    \choice[correct]{$\frac{f(a) - f(a-h)}{a-(a-h)}$}
  \end{selectAll}
\end{problem}

\begin{problem}
   Let $f(x) = 3x-1$.  Zoom in on the curve around $a = -2$ so that $h
   = 0.1$.  Use one of the formulations in the problem above to
   approximate the slope of the curve.  The slope of the curve at $a =
   -2$ is approximately\dots \begin{prompt}$\answer{3}$\end{prompt}
\end{problem}

\begin{problem}
   Repeat the previous problem for $f(x) = x^2 - 1$, $a = 0$, and $h =
   0.2$.  Choose a formulation that will give you a positive answer
   for the slope.  The (positive) slope of the curve at $a = 0$ is
   approximately\dots\begin{prompt} $\answer{0.2}$\end{prompt}
\end{problem}


\begin{problem}
   Zoom in on the curve $f(x) = x^2 - 1$ near $x=0$ again.  By looking
   at the graph, what is your best guess for the actual slope of the
   curve at zero?
   \begin{multipleChoice}
     \choice{impossible to say}
     \choice[correct]{zero}
     \choice{one}
     \choice{infinity}
   \end{multipleChoice}
\end{problem}

%\input{../leveledQuestions.tex}


\end{document}
