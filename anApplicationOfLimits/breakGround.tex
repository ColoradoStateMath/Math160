\documentclass{ximera}

\input{../preamble.tex}

\outcome{Consider limits as behavior nearer and nearer to a point.}


\title[Break-Ground:]{Limits and velocity}

\begin{document}
\begin{abstract}
Two young mathematicians discuss limits and instantaneous velocity.
\end{abstract}
\maketitle

Check out this dialogue between two calculus students (based on a true
story):

\begin{dialogue}
\item[Devyn] Hey Riley, I've been thinking about limits.
\item[Riley] That is awesome.
\item[Devyn] I know! You know limits remind me of something\dots
  How a GPS or a phone computes velocity!
\item[Riley] Huh.  A GPS can calculate our location.   Then, to compute velocity 
from position, it must look at
  \[
  \frac{\text{change in position}}{\text{change in time}}
  \]
\item[Devyn] And then we study this as the change in time gets closer
  and closer to zero.
\item[Riley] Just like with limits at zero, we can study something by
  looking \textbf{near} a point, but \textbf{not exactly at} a point.
\item[Devyn] O.M.G.\ Life's a rich tapestry.
\item[Riley] Poet, you know it.
\end{dialogue}



Suppose you take a road trip from Columbus Ohio to Urbana-Champaign
Illinois. Moreover, suppose your position is modeled by
\[
s(t) = 36t^2 -4.8t^3 \qquad\text{(miles West of Columbus)} %% note the model is wrong
\]
where $t$ is measured in hours and runs from $0$ to $5$ hours. 


\begin{problem}
  What is the average velocity for the entire trip?
  \begin{hint}
    Remember, 
    \[
    \text{change in distance} = \text{rate}\cdot\text{change in time}.
    \]
  \end{hint}
  \begin{hint}
    So, 
    \[
    \frac{\Delta\text{distance}}{\Delta\text{time}} = \text{rate}.
    \]
  \end{hint}
  \begin{hint}
    So, 
    \[
    \frac{\Delta\text{distance}}{\Delta\text{time}} = \frac{\answer[given]{300}}{\answer[given]{5}}.
    \]
  \end{hint}
  \begin{prompt}
    The average velocity is \answer{60} miles per hour.
  \end{prompt}
\end{problem}


\begin{problem}
  Use a calculator to estimate the instantaneous velocity at $t=2$.
  \begin{hint}
    Remember, 
    \[
    \text{change in distance} = \text{rate}\cdot\text{change in time}.
    \]
  \end{hint}
  \begin{hint}
    So, 
    \[
    \frac{\Delta\text{distance}}{\Delta\text{time}} = \text{rate}.
    \]
  \end{hint}
  \begin{hint}
    Compute
    \[
    \frac{36(2+\Delta t)^2 -4.8(2+\Delta t)^3 -\left(36\cdot 2^2 -4.8\cdot 2^3\right) }{\Delta t}
    \]
    for smaller, and smaller values of $\Delta t$.
  \end{hint}
  \begin{prompt}
    The instantaneous velocity, (rounded to the nearest tenth) is \answer{86.4} miles per hour.
  \end{prompt}
\end{problem}


\begin{problem}
  Considering the work above, when we want to compute instantaneous
  velocity, we need to compute
  \[
  \frac{\text{change in position}}{\text{change in time}}
  \]
  when (choose all that apply):
 \begin{selectAll}
    \choice{The ``change in time'' is zero.}
    \choice[correct]{The ``change in time'' gets closer and closer to zero.}
    \choice[correct]{The ``change in time'' approaches zero.}
    \choice{The ``change in time'' is near zero.}
    \choice[correct]{The ``change in time'' goes to zero.}
 \end{selectAll}
\end{problem}


Computing average velocities for smaller, and smaller, values of
$\Delta t$ as we did above is tedious. Nevertheless, this is exactly
how a GPS determines velocity from position! To avoid these tedious
calculations, we would really like to have a formula.

%On the other hand, we are
%human beings and have better things to do than just compute all day
%long. What would really help us out with problems like these is a formula.



\input{../leveledQuestions.tex}


\end{document}
