\documentclass{ximera}

\input{../preamble.tex}

\outcome{Use ``shortcut'' rules to find derivatives}


\title[Break-Ground:]{Patterns in derivatives}

\begin{document}
\begin{abstract}
Two young mathematicians think about ``short cuts'' for differentiation. 
\end{abstract}
\maketitle

Check out this dialogue between two calculus students (based on a true
story):

\begin{dialogue}
\item[Devyn] I hate the limit definition of derivative.  I wish there
  were a shorter way.
\item[Riley] I think I might have found a pattern for taking
  derivatives.
\item[Devyn] Really? I love patterns!
\item[Riley] I know! Check this out, I've made a chart
{  \renewcommand*{\arraystretch}{1.3}
  \[
  \begin{array}{c|c}
    f(x) & f'(x)\\ \hline
    x^2 & 2\cdot x^1\\
    x^3 & 3\cdot x^2\\
    x^4 & 4\cdot x^3
  \end{array}
  \]
  }
  So maybe if we have a function
  \[
  f(x) = x^n\quad\text{then}\qquad f'(x) = n\cdot x^{n-1}.
  \]
\item[Devyn] Hmmm does it work with square roots?
\item[Riley] Oh that's right, a square root is a power, just write
  \[
  f(x) = \sqrt{x} = x^{1/2}.
  \]
  So a square root is of the form $x^n$.
\item[Devyn] Let's check it. If $f(x) = \sqrt{x}$,
  \begin{align*}
    f'(x) &= \lim_{h\to 0} \frac{\sqrt{x+h} -\sqrt{x}}{h}\\
    &= \lim_{h\to 0} \left(\frac{\sqrt{x+h} - \sqrt{x}}{h}\cdot \frac{\sqrt{x+h} + \sqrt{x}}{\sqrt{x+h} + \sqrt{x}}\right)\\
    &= \lim_{h\to 0} \frac{x+h - x}{h(\sqrt{x+h} + \sqrt{x})}\\
    &= \lim_{h\to 0} \frac{h}{h(\sqrt{x+h} + \sqrt{x})}\\
    &= \lim_{h\to 0} \frac{1}{\sqrt{x+h} + \sqrt{x}}\\
    &= \frac{1}{\sqrt{x} + \sqrt{x}}\\
    &= \frac{1}{2\sqrt{x}}\\
    &= \frac{1}{2}\cdot x^{-1/2}.
  \end{align*}
\item[Riley] Holy Cat Fur! It works! In this case $f'(x) = n\cdot
  x^{n-1}$.
  \item[Devyn] I wonder if it \textit{always} works? If so I want to
    know \textit{why} it works! I wonder what other patterns we can
    find?
\end{dialogue}

The pattern
\[
  \text{if} \qquad f(x) = x^n\quad\text{then}\qquad f'(x) = n\cdot x^{n-1}
\]
holds whenever $n$ is a constant. Explaining why it works in
generality will take some time. For now, let's see if we can use the
problem to squash some derivatives with ease.

\begin{problem}
  Using the pattern found above, compute:
  \[
  \ddx x^{101} \begin{prompt}= \answer{101 x^{100}}\end{prompt}
  \]
\end{problem}

\begin{problem}
  Using the pattern found above, compute:
  \[
  \ddx \frac{1}{x^{77}} \begin{prompt}= \answer{-77 x^{-78}}\end{prompt}
  \]
\end{problem}


\begin{problem}
  Using the pattern found above, compute:
  \[
  \ddx \sqrt[5]{x} \begin{prompt}= \answer{x^{-4/5}/5}\end{prompt}
  \]
\end{problem}

\begin{problem}
  Using the pattern found above, compute:
  \[
  \ddx x^e  \begin{prompt}= \answer{e x^{e-1}}\end{prompt}
  \]
\end{problem}

\input{../leveledQuestions.tex}

\end{document}
