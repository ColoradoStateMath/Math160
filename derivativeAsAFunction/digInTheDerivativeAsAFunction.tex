\documentclass{ximera}

\input{../preamble.tex}

\outcome{Understand the derivative as a function related to the original
  definition of a function.}
\outcome{Find the derivative function using the limit definition.}
\outcome{Relate the derivative function to the derivative at a point.}
\outcome{Relate the graph of the function to the graph of its derivative.}




\title[Dig-in:]{The derivative as a function}

\begin{document}
\begin{abstract}
Here we study the derivative of a function, as a function, in its own
right.
\end{abstract}
\maketitle

\section{The derivative of a function, as a function}


We know that to find the derivative of a function at a point $x=a$ we
write
\[
f'(a) = \lim_{h\to 0}\frac{f(a+h)-f(a)}{h}.
\]
However, if we replace the given number $a$ with a variable $x$, we now
have
\[
f'(x) = \lim_{h\to 0}\frac{f(x+h)-f(x)}{h}.
\]
This tells us the instantaneous rate of change at any given point $x$.
\begin{warning}
  The notation:
  \begin{quote}
  $f'(a)$ means take the derivative of $f$ first, then evaluate at
    $x=a$.
  \end{quote}
  In other words, given $f$ a function of $x$
  \[
  f'(a) = \eval{\ddx f(x)}_{x=a}.
  \]
\end{warning}

\begin{question}
Suppose you are given a polynomial $f(x)$ and you are asked to compute $f'(2)$. What is the most efficient way to answer the question?
  \begin{multipleChoice}
  \choice[correct]{Compute $f'(2)$ directly from the definition of \\
  derivative of a function at a point}
  \choice{Compute the derivative as a function $f'(x)$, and then evaluate $f'(x)$ at $x=2$}
  \end{multipleChoice}
  \begin{question}
  Suppose you are given a polynomial $f(x)$ and you are asked to compute $f'(-2)$, $f'(0)$, and $f'(2)$. What is the most efficient way to answer the question?
    \begin{multipleChoice}
      \choice{Compute $f'(-2)$, $f'(0)$, and $f'(2)$ directly from the\\
      definition of derivative of a function at a point}
      \choice[correct]{Compute the derivative of $f$ as a function $f'(x)$, \\
      and then evaluate $f'(x)$ at $x=-2$, $0$, and $-2$}
    \end{multipleChoice}
    \begin{feedback}
    Typically, a limit computation takes longer than evaluation,
    so when you need to find that derivative of a function at multiple points,
    it is often quicker to compute the derivative as a function,
    and then evaluate at the particular points.
    \end{feedback}
  \end{question}

\end{question}


Given a function $f$ from the real numbers to the real numbers, the
derivative $f'$ is also a function from the real numbers to the real
numbers. Understanding the relationship between the \textit{functions}
$f$ and $f'$ helps us understand any situation (real or imagined)
involving changing values. 

\begin{question}
  Let $f(x) = 3x+2$. What is $f'(-1)$?
  \begin{multipleChoice}
    \choice{$f'(-1) = 0$ because $f'(3)$ is a number, and a number corresponds to a horizontal line,\\ which has a slope of zero.}
    \choice[correct]{$f'(-1) = 3$ because $y=f(x)$ is a straight line with slope $3$.}
    \choice{We cannot solve this problem yet.}
  \end{multipleChoice}
\end{question}


\begin{question}
  Here we see the graph of $f'$. 
  \begin{image}
    \begin{tikzpicture}
      \begin{axis}[
          xmin=-2,xmax=2,ymin=-8,ymax=8,
          axis lines=center,
          ticks=none,
          width=6in,
          height=3in,
          every axis y label/.style={at=(current axis.above origin),anchor=south},
          every axis x label/.style={at=(current axis.right of origin),anchor=west},
        ]        
        %\addplot [very thick,dashed, penColor,smooth, domain=(-2:2)] {x^3+.3*x^2-2*x)};
        \addplot [very thick,penColor,smooth, domain=(-2:2)] {3*x^2+2*.3*x-2)};
      \end{axis}
    \end{tikzpicture}
  \end{image}
  Describe $y=f(x)$ when $f'$ is positive. Describe $y=f(x)$ when $f'$
  is negative.
  \begin{prompt}
    When $f'$ is positive, $y=f(x)$ is \wordChoice{\choice{positive}\choice[correct]{increasing}\choice{negative}\choice{decreasing}}.
    When $f'$ is negative, $y=f(x)$ is \wordChoice{\choice{positive}\choice{increasing}\choice{negative}\choice[correct]{decreasing}}
  \end{prompt}
  \begin{question}
    Which of the following graphs could be $y = f(x)$?
     \begin{multipleChoice}
       \choice{\begin{tikzpicture}[framed,scale=1,baseline=3ex]
           \begin{axis}[
               xmin=-2,xmax=2,ymin=-8,ymax=8,
               axis lines=center,
               ticks=none,
               width=2in,
               height=1in,
               every axis y label/.style={at=(current axis.above origin),anchor=south},
               every axis x label/.style={at=(current axis.right of origin),anchor=west},
             ]        
             \addplot [very thick,penColor,smooth, domain=(-2:2)] {3*x^2+2*.3*x-2)};
           \end{axis}
       \end{tikzpicture}}
       \choice[correct]{\begin{tikzpicture}[framed,scale=1,baseline=3ex]
           \begin{axis}[
               xmin=-2,xmax=2,ymin=-8,ymax=8,
               axis lines=center,
               ticks=none,
               width=2in,
               height=1in,
               every axis y label/.style={at=(current axis.above origin),anchor=south},
               every axis x label/.style={at=(current axis.right of origin),anchor=west},
             ]        
             \addplot [very thick,penColor,smooth, domain=(-2:2)] {x^3+.3*x^2-2*x)};
           \end{axis}
       \end{tikzpicture}}
       \choice{\begin{tikzpicture}[framed,scale=1,baseline=3ex]
           \begin{axis}[
               xmin=-2,xmax=2,ymin=-8,ymax=8,
               axis lines=center,
               ticks=none,
               width=2in,
               height=1in,
               every axis y label/.style={at=(current axis.above origin),anchor=south},
               every axis x label/.style={at=(current axis.right of origin),anchor=west},
             ]        
             \addplot [very thick,penColor,smooth, domain=(-2:2)] {6*x+2*.3)};
           \end{axis}
       \end{tikzpicture}}
     \end{multipleChoice}
  \end{question}
\end{question}



\section{The derivative as a function on functions}

While writing $f'$ is viewing the derivative of $f$ as a function in
its own right, the derivative itself
\[
\ddx
\]
is in fact a function that maps functions to functions,
\begin{align*}
  \ddx x^2 &= 2x\\
  \ddx f(x) &= f'(x).
\end{align*}

\begin{question}
  As a function, is
  \[
  \ddx
  \]
  one-to-one? 
  \begin{hint}
  Recall that a function is one-to-one if every input has a distinct output, i.e. two inputs have the same output only when they are the same input.
  \end{hint}
  \begin{multipleChoice}
    \choice{yes}
    \choice[correct]{no}
  \end{multipleChoice}
  \begin{feedback}
    Many different functions share the same derivative since the
    derivative records only the slope of the tangent line and not
    the value, or height.
  \end{feedback}
\end{question}

\end{document}

