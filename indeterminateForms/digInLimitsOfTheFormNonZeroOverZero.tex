\documentclass{ximera}

\outcome{Calculate limits of the form number over zero.}
\outcome{Identify determinate and indeterminate forms.}
\outcome{Distinguish between determinate and indeterminate forms.}

\input{../preamble.tex}

\title[Dig-In:]{Limits of the form nonzero over zero}

\begin{document}
\begin{abstract}
  We want to solve limits that have the form nonzero over zero.
\end{abstract}

\maketitle

Let's cut to the chase:

\begin{definition}
  A limit
  \[
  \lim_{x\to a} \frac{f(x)}{g(x)}
  \]
  is said to be of the form \numOverZero\ if
  \[
  \lim_{x\to a} f(x) = k\qquad\text{and}\qquad \lim_{x\to a} g(x) =0.
  \]
  where $k$ is some nonzero constant.
\end{definition}

\begin{question}
  Which of the following limits are of the form \numOverZero?
  \begin{selectAll}
    \choice[correct]{$\displaystyle\lim_{x\to -1} \frac{1}{(x+1)^2}$}
    \choice{$\displaystyle\lim_{x\to 2}\frac{x^2-3x+2}{x-2}$}
    \choice{$\displaystyle\lim_{x\to 0}\frac{\sin(x)}{x}$}
    \choice[correct]{$\displaystyle\lim_{x\to 2}\frac{x^2-3x-2}{x-2}$}
  \end{selectAll}
\end{question}


Let's see what is going on with limits of the form \numOverZero.
Consider the function
  \[
  f(x) = \frac{1}{(x+1)^2}.
  \]
While the $\lim_{x\to -1} f(x)$ does not exist, something can still be
said. First note that
\[
\lim_{x\to -1} \frac{1}{(x+1)^2}\qquad\text{is of the form }\numOverZero
\]
as
\[
\lim_{x\to -1} 1 = 1 \qquad\text{and}\qquad \lim_{x\to -1}(x+1)^2 = 0.
\]
Moreover, as $x$ approaches $-1$:
\begin{itemize}
\item The numerator is positive.
\item The denominator approaches zero and is positive.
\end{itemize}
Hence
\[
\lim_{x\to -1} \frac{1}{(x+1)^2}
\]
will become arbitrary large, as we can see in the next graph.
\begin{image}
\begin{tikzpicture}
	\begin{axis}[
            domain=-2:1,
            ymax=100,
            samples=100,
            axis lines =middle, xlabel=$x$, ylabel=$y$,
            every axis y label/.style={at=(current axis.above origin),anchor=south},
            every axis x label/.style={at=(current axis.right of origin),anchor=west}
          ]
	  \addplot [very thick, penColor, smooth, domain=(-2:-1.1)] {1/(x+1)^2};
          \addplot [very thick, penColor, smooth, domain=(-.9:1)] {1/(x+1)^2};
          \addplot [textColor, dashed] plot coordinates {(-1,0) (-1,100)};
        \end{axis}
\end{tikzpicture}
\end{image}

We are now ready for our next definition.

\begin{definition}
If $f(x)$ grows arbitrarily large as $x$ approaches $a$, we write
\[
\lim_{x\to a} f(x) = \infty
\]
and say that the limit of $f(x)$ \dfn{approaches infinity} as $x$
goes to $a$.

If $|f(x)|$ grows arbitrarily large as $x$ approaches $a$ and $f(x)$ is
negative, we write
\[
\lim_{x\to a} f(x) = -\infty
\]
and say that the limit of $f(x)$ \textbf{approaches negative infinity}
as $x$ goes to $a$.
\end{definition}

Let's consider a few more examples.

\begin{example}
  Compute:
  \[
  \lim_{x\to -2} \frac{3x}{(x+2)^4}
  \]
  \begin{explanation}
    First let's look at the form of this limit, we do this by taking the limits of both the numerator and denominator:
    \[
    \lim_{x\to -2} 3x = \answer[given]{-6}\qquad\text{and}\lim_{x\to-2}\left((x+2)^4\right) = 0
    \]
    so this limit is of the form \numOverZero.  As $x$ approaches $-2$:
    \begin{itemize}
    \item The numerator is a \wordChoice{\choice[correct]{positive}\choice{negative}} number. 
    \item The denominator is \wordChoice{\choice[correct]{positive}\choice{negative}} and is approaching zero.
    \end{itemize}
    This means that
    \[
    \lim_{x\to -2} \frac{3x}{(x+2)^4} = -\infty.
    \]
  \end{explanation}
\end{example}


\begin{example}
  Compute:
  \[
  \lim_{x\to 3^+} \frac{x^2-9x+14}{x^2-5x+6}
  \]
  \begin{explanation}
    First let's look at the form of this limit, which we do by taking the limits of both the numerator and denominator.
    \[
    \lim_{x\to 3^+} \left(x^2-9x+14\right) = \answer[given]{-4}\qquad\text{and}\lim_{x\to3^+}\left(x^2-5x+6\right) = 0
    \]
    This limit is of the form \numOverZero. Next, we should factor the numerator and denominator to see if we can simplify the problem at all. 
    \begin{align*}
      \lim_{x\to 3^+}\frac{x^2-9x+14}{x^2-5x+6} &= \lim_{x\to 3^+}\frac{(x-2)(x-7)}{(x-2)(x-3)}\\
      &= \lim_{x\to 3^+}\frac{x-7}{x-3}\\
    \end{align*}
    Canceling a factor of $x-2$ from the numerator and denominator
    means we can more easily check the behavior of this limit.  As $x$
    approaches $3$ from the right:
    \begin{itemize}
    \item The numerator is a \wordChoice{\choice{positive}\choice[correct]{negative}} number. 
    \item The denominator is \wordChoice{\choice[correct]{positive}\choice{negative}} and approaching zero.
    \end{itemize}
    This means that
    \[
    \lim_{x\to 3^+} \frac{x^2-9x+14}{x^2-5x+6} = -\infty.
    \]
   \end{explanation}
\end{example}

Here is our final example.

\begin{example}
  Compute:
  \[
  \lim_{x\to 3} \frac{x^2-9x+14}{x^2-5x+6}
  \]
  \begin{explanation}
    We've already considered part of this example, but now we consider the two-sided limit. We already know that
    \[
    \lim_{x\to 3} \frac{x^2-9x+14}{x^2-5x+6} = \lim_{x\to
      3}\frac{x-7}{x-3},
    \]
    and that this limit is of the form \numOverZero.
    We also know that as $x$ approaches $3$ from the right,
    \begin{itemize}
    \item The numerator is a negative number. 
    \item The denominator is positive and approaching zero.
    \end{itemize}
    Hence our function is approaching $-\infty$ from the right.
    
    As $x$ approaches $3$ from the left,
    \begin{itemize}
    \item The numerator is negative.
    \item The denominator is negative and approaching zero.
    \end{itemize}
    Hence our function is approaching $\infty$ from the left.
    This means
    \[
    \lim_{x\to 3} \frac{x^2-9x+14}{x^2-5x+6} = \answer[given]{DNE}.
    \]
  \end{explanation}
\end{example}

Some people worry that the mathematicians are passing into mysticism
when we talk about infinity and negative infinity. However, when we write
\[
\lim_{x\to a} f(x) = \infty \qquad\text{and}\qquad \lim_{x\to a} g(x) = -\infty
\]
all we mean is that as $x$ approaches $a$, $f(x)$ becomes arbitrarily
large and $|g(x)|$ becomes arbitrarily large, with $g(x)$ taking
negative values.
\end{document}
