\documentclass{ximera}

\input{../preamble.tex}

\title[Dig-In]{Maximums and minimums}



\outcome{Define a critical point.}
\outcome{Find critical points.}
\outcome{Define absolute maximum and absolute minimum.}
\outcome{Find the absolute max or min of a continuous function on a closed interval.}
\outcome{Define local maximum and local minimum.}
\outcome{Compare and contrast local and absolute maxima and minima.}
\outcome{Identify situations in which an absolute maximum or minimum is guaranteed.}
\outcome{Classify critical points.}
\outcome{State the First Derivative Test.}
\outcome{Apply the First Derivative Test.}
\outcome{State the Second Derivative Test.}
\outcome{Apply the Second Derivative Test.}
\outcome{Define inflection points.}
\outcome{Find inflection points.}
  


\begin{document}
\begin{abstract}
We use derivatives to help locate extrema.  
\end{abstract}
\maketitle


Whether we are interested in a function as a purely mathematical
object or in connection with some application to the real world, it is
often useful to know what the graph of the function looks like. We can
obtain a good picture of the graph using certain crucial information
provided by derivatives of the function.

\section{Extrema}

Local \textit{extrema} on a function are points on the graph where the
$y$-coordinate is larger (or smaller) than all other $y$-coordinates
on the graph at points ``close to'' $(x,y)$. 

\begin{definition}\hfil\index{maximum/minimum!local}
\begin{enumerate}
\item A function $f$ has a \dfn{local maximum} at $x=a$, if $f(a)\ge
  f(x)$ for every $x$ near $a$.
\item A function $f$ has a \dfn{local minimum} at $x=a$, if $f(a)\le
  f(x)$ for every $x$ near $a$.
\end{enumerate}
A \dfn{local extremum}\index{extremum!local} is either a local
maximum or a local minimum.
\end{definition}

\begin{problem}
  True or false: ``All absolute extrema are also local extrema.''
  \begin{multipleChoice}
    \choice[correct]{true}
    \choice{false}
  \end{multipleChoice}
  \begin{feedback}
    All global extrema are local extrema.
  \end{feedback}
\end{problem}

Local maximum and minimum points are quite distinctive on the graph of
a function, and are therefore useful in understanding the shape of the
graph. In many applied problems we want to find the largest or
smallest value that a function achieves (for example, we might want
to find the minimum cost at which some task can be performed) and so
identifying maximum and minimum points will be useful for applied
problems as well.



\section{Critical points}


If $(x,f(x))$ is a point where $f$ reaches a local maximum or minimum,
and if the derivative of $f$ exists at $x$, then the graph has a
tangent line and the tangent line must be horizontal. This is
important enough to state as a theorem, though we will not prove it.

\begin{theorem}[Fermat's Theorem]\index{Fermat's Theorem}\label{theorem:fermat}
If $f$ has a local extremum at $x=a$ and $f$ is differentiable
at $a$, then $f'(a)=0$.
\end{theorem}

\begin{problem}
  Does Fermat's Theorem say that if $f'(a) = 0$, then $f$ has a local
  extrema at $x=a$?
  \begin{multipleChoice}
    \choice{yes}
    \choice[correct]{no}
  \end{multipleChoice}
  \begin{feedback}
    Consider $f(x) = x^3$, $f'(0) = 0$, but $f$ does not have a local
    maximum or minimum at $x=0$.
  \end{feedback}
\end{problem}


Fermat's Theorem says that the only points at which a function can
have a local maximum or minimum are points at which the derivative is
zero, consider the plots of $f(x) = x^3-4x^2+3x$ and $f'(x) =
3x^2-8x+3$,
\begin{image}
\begin{tikzpicture}
	\begin{axis}[
            domain=-3:3,
            ymax=3,
            ymin=-3,
            %samples=100,
            axis lines =middle, xlabel=$x$, ylabel=$y$,
            every axis y label/.style={at=(current axis.above origin),anchor=south},
            every axis x label/.style={at=(current axis.right of origin),anchor=west}
          ]
          \addplot [dashed, textColor, smooth] plot coordinates {(.451,0) (.451,.631)}; %% {.451};
          \addplot [dashed, textColor, smooth] plot coordinates {(2.215,-2.113) (2.215,0)}; %% axis{2.215};
          \addplot [very thick, penColor2, smooth] {3*x^2-8*x+3};
          \addplot [very thick, penColor, smooth] {x^3-4*x^2+3*x};
          \node at (axis cs:2.5,-2) [anchor=west] {\color{penColor}$f$};  
          \node at (axis cs:.2,2) [anchor=west] {\color{penColor2}$f'$};
          \addplot[color=penColor2,fill=penColor2,only marks,mark=*] coordinates{(.451,0)};  %% closed hole
          \addplot[color=penColor2,fill=penColor2,only marks,mark=*] coordinates{(2.215,0)};  %% closed hole
          \addplot[color=penColor,fill=penColor,only marks,mark=*] coordinates{(.451,.631)};  %% closed hole
          \addplot[color=penColor,fill=penColor,only marks,mark=*] coordinates{(2.215,-2.113)};  %% closed hole
        \end{axis}
\end{tikzpicture}
%% \caption{A plot of $f(x) = x^3-4x^2+3x$ and $f'(x) = 3x^2-8x+3$.}
%% \label{figure:x^3-4x^2+3x}
\end{image}
or the derivative is undefined, as in the plot of $f(x) = x^{2/3}$ and $f'(x) = \frac{2}{3x^{1/3}}$:
\begin{image}
\begin{tikzpicture}
	\begin{axis}[
            domain=-3:3,
            ymax=2,
            ymin=-2,
            axis lines =middle, xlabel=$x$, ylabel=$y$,
            every axis y label/.style={at=(current axis.above origin),anchor=south},
            every axis x label/.style={at=(current axis.right of origin),anchor=west}
          ]
          \addplot [very thick, penColor2, samples=100, smooth,domain=(-3:-.01)] {-(2/3)*abs(x)^(-1/3)};
          \addplot [very thick, penColor2, samples=100, smooth,domain=(.01:3)] {(2/3)*abs(x)^(-1/3)};
          \addplot [very thick, penColor, smooth,domain=(-3:-.01)] {abs(x)^(2/3)};
          \addplot [very thick, penColor, smooth,domain=(.01:3)] {x^(2/3)};         
          \node at (axis cs:-2,1.7) [anchor=west] {\color{penColor}$f$};  
          \node at (axis cs:2,.7) [anchor=west] {\color{penColor2}$f'$};
        \end{axis}
\end{tikzpicture}
%% \caption{A plot of $f(x) = x^{2/3}$ and $f'(x) = \frac{2}{3x^{1/3}}$.}
%% \label{figure:x^{2/3}}
\end{image}
This brings us to our next definition.

\begin{definition}\index{critical point}
  A function has a \dfn{critical point} at $x=a$ if 
  \[
  f'(a) = 0\qquad\text{or}\qquad \text{$f'(a)$ does not exist.}
  \]
\end{definition}

\begin{warning} 
When looking for local maximum and minimum points, you are likely to
make two sorts of mistakes: 
\begin{itemize}
\item You may forget that a maximum or minimum can occur where the
  derivative does not exist, and so forget to check whether the
  derivative exists everywhere. 
\item You might assume that any place that the derivative is zero is a
  local maximum or minimum point, but this is not true, consider the
  plots of $f(x) = x^3$ and $f'(x) = 3x^2$.
\begin{image}
\begin{tikzpicture}
	\begin{axis}[
            domain=-3:3,
            ymax=3,
            ymin=-3,
            axis lines =middle, xlabel=$x$, ylabel=$y$,
            every axis y label/.style={at=(current axis.above origin),anchor=south},
            every axis x label/.style={at=(current axis.right of origin),anchor=west}
          ]
          \addplot [very thick, penColor2, smooth] {3*x^2};
          \addplot [very thick, penColor, smooth] {x^3};         
          \node at (axis cs:1,.9) [anchor=west] {\color{penColor}$f$};  
          \node at (axis cs:-.5,1) [anchor=west] {\color{penColor2}$f'$};
        \end{axis}
\end{tikzpicture}
%% \caption{A plot of $f(x) = x^3$ and $f'(x) = 3x^2$. While $f'(0)=0$,
%%   there is neither a maximum nor minimum at $(0,f(0))$.}
%% \label{figure:x^3}
\end{image}
While $f'(0)=0$, there is neither a maximum nor minimum at $(0,f(0))$.
\end{itemize}
\end{warning}



Since the derivative is zero or undefined at both local maximum and
local minimum points, we need a way to determine which, if either,
actually occurs. The most elementary approach is to test directly
whether the $y$ coordinates near the potential maximum or minimum are
above or below the $y$ coordinate at the point of interest. 

It is not always easy to compute the value of a function at a
particular point. The task is made easier by the availability of
calculators and computers, but they have their own drawbacks: they do
not always allow us to distinguish between values that are very close
together. Nevertheless, because this method is conceptually simple and
sometimes easy to perform, you should always consider it.




\begin{example}
Find all local maximum and minimum points for the function 
$f(x)=x^3-x$. 
\begin{explanation} 
Write
\[
\ddx f(x)=\answer[given]{3x^2-1}.
\] 
This is defined everywhere and is zero at $x=\pm \sqrt{3}/3$. Looking
first at $x=\sqrt{3}/3$, we see that 
\[
f(\sqrt{3}/3)=\answer[given]{-2\sqrt{3}/9}.
\] 
Now we test two points on either side of $x=\sqrt{3}/3$, making sure
that neither is farther away than the nearest critical point; since
$\sqrt{3}<3$, $\sqrt{3}/3<1$ and we can use $x=0$ and $x=1$. Since
\[
f(0)=0>-2\sqrt{3}/9\qquad\text{and}\qquad f(1)=0>-2\sqrt{3}/9,
\] 
there must be a local minimum at $x=\answer[given]{\sqrt{3}/3}$.

For $x=-\sqrt{3}/3$, we see that $f(-\sqrt{3}/3)=2\sqrt{3}/9$. This
time we can use $x=0$ and $x=-1$, and we find that $f(-1)=f(0)=0<
2\sqrt{3}/9$, so there must be a local maximum at
$x=\answer[given]{-\sqrt{3}/3}$, see the plot below:
\begin{image}
\begin{tikzpicture}
	\begin{axis}[
            domain=-2:2,
            ymax=2,
            ymin=-2,
            %samples=100,
            axis lines =middle, xlabel=$x$, ylabel=$y$,
            every axis y label/.style={at=(current axis.above origin),anchor=south},
            every axis x label/.style={at=(current axis.right of origin),anchor=west}
          ]
          \addplot [dashed, textColor, smooth] plot coordinates {(-.577,0) (-.577,.385)}; %% {.451};
          \addplot [dashed, textColor, smooth] plot coordinates {(.577,-.385) (.577,0)}; %% axis{2.215};

          \addplot [very thick, penColor2, smooth] {3*x^2-1};
          \addplot [very thick, penColor, smooth] {x^3-x};

          \node at (axis cs:1.2,.3) [anchor=west] {\color{penColor}$f$};  
          \node at (axis cs:-.75,1) [anchor=west] {\color{penColor2}$f'$};

          \addplot[color=penColor2,fill=penColor2,only marks,mark=*] coordinates{(-.577,0)};  %% closed hole
          \addplot[color=penColor2,fill=penColor2,only marks,mark=*] coordinates{(.577,0)};  %% closed hole
          \addplot[color=penColor,fill=penColor,only marks,mark=*] coordinates{(-.577,.385)};  %% closed hole
          \addplot[color=penColor,fill=penColor,only marks,mark=*] coordinates{(.577,-.385)};  %% closed hole
        \end{axis}
\end{tikzpicture}
%%\caption{A plot of $f(x) = x^3-x$ and $f'(x) = 3x^2-1$.}
%%\label{figure:x^3-x}
\end{image}
\end{explanation}
\end{example}






\section{The first derivative test}

The method of the previous section for deciding whether there is a
local maximum or minimum at a critical point by testing ``near-by''
points is not always convenient. Instead, since we have already had to
compute the derivative to find the critical points, we can use
information about the derivative to decide. Recall that
\begin{itemize}
\item If $f'(x) >0$ on an interval, then $f$ is increasing on that interval.
\item If $f'(x) <0$ on an interval, then $f$ is decreasing on that interval.
\end{itemize}

So how exactly does the derivative tell us whether there is a maximum,
minimum, or neither at a point? Use the \textit{first derivative test}.

\begin{theorem}[First Derivative Test]\index{first derivative test}\label{T:fdt}
Suppose that $f$ is continuous on an interval, and that $f'(a)=0$ for
some value of $a$ in that interval.
\begin{itemize}
\item If $f'(x)>0$ to the left of $a$ and $f'(x)<0$ to the right of
  $a$, then $f(a)$ is a local maximum.
\item If $f'(x)<0$ to the left of $a$ and $f'(x)>0$ to the right of
  $a$, then $f(a)$ is a local minimum.
\item If $f'(x)$ has the same sign to the left and right of $a$,
  then $f(a)$ is not a local extremum.
\end{itemize}
\end{theorem}

\begin{example}\label{E:localextrema}
Consider the function 
\[
f(x) = \frac{x^4}{4}+\frac{x^3}{3}-x^2
\]
Find the intervals on which $f$ is increasing and decreasing and
identify the local extrema of $f$.


\begin{explanation}
Start by computing
\[
\ddx f(x) = \answer[given]{x^3+x^2-2x}.
\]
Now we need to find when this function is positive and when it is
negative. To do this, solve 
\[
f'(x) = \answer[given]{x^3+x^2-2x} =0.
\]
Factor $f'(x)$
\begin{align*}
f'(x) &= \answer[given]{x^3+x^2-2x} \\
&=x(\answer[given]{x^2+x-2})\\
&=x(x+2)\answer[given]{(x-1)}.
\end{align*}
So the critical points (when $f'(x)=0$) are when $x=-2$, $x=0$, and
$x=1$. Now we can check points \textbf{between} the critical points to find
when $f'(x)$ is increasing and decreasing:
\begin{align*}
  f'(-3)&=\answer[given]{-12},\\
  f'(.5)&=\answer[given]{-0.625},\\
  f'(-1)&=\answer[given]{2},\\
  f'(2)&=\answer[given]{8}.
\end{align*}
From this we can make a sign table:

\begin{image}
\begin{tikzpicture}
	\begin{axis}[
            trim axis left,
            scale only axis,
            domain=-3:3,
            ymax=2,
            ymin=-2,
            axis lines=none,
            height=3cm, %% Hard coded height! 
            width=\textwidth, %% width
          ]
          %\addplot [draw=none, fill=fill1, domain=(-3:-2)] {2} \closedcycle;
          %\addplot [draw=none, fill=fill2, domain=(-2:0)] {2} \closedcycle;
          %\addplot [draw=none, fill=fill1, domain=(0:1)] {2} \closedcycle;
          %\addplot [draw=none, fill=fill2, domain=(1:3)] {2} \closedcycle;
          
          \addplot [->,textColor] plot coordinates {(-3,0) (3,0)}; %% axis{0};

          \addplot [->,ultra thick,textColor,shorten <=2pt,shorten >=2pt] plot coordinates {(-3,1.5) (-2,.5)}; %% decreasing
          \addplot [->,ultra thick,textColor,shorten <=2pt,shorten >=2pt] plot coordinates {(-2,.5) (0,1.5)}; %% increasing
          \addplot [->,ultra thick,textColor,shorten <=2pt,shorten >=2pt] plot coordinates {(0,1.5) (1,.5)}; %% decreasing
          \addplot [->,ultra thick,textColor,shorten <=2pt,shorten >=2pt] plot coordinates {(1,.5) (3,1.5)}; %% increasing
          
          \addplot [dashed, textColor] plot coordinates {(-2,0) (-2,2)};
          \addplot [dashed, textColor] plot coordinates {(0,0) (0,2)};
          \addplot [dashed, textColor] plot coordinates {(1,0) (1,2)};
          
          \node at (axis cs:-2,0) [anchor=north,textColor] {\footnotesize$-2$};
          \node at (axis cs:0,0) [anchor=north,textColor] {\footnotesize$0$};
          \node at (axis cs:1,0) [anchor=north,textColor] {\footnotesize$1$};

          \node at (axis cs:-2.5,-.7) [textColor] {\footnotesize$f'(x)<0$};
          \node at (axis cs:.5,-.7) [textColor] {\footnotesize$f'(x)<0$};
          \node at (axis cs:-1,-.7) [textColor] {\footnotesize$f'(x)>0$};
          \node at (axis cs:2,-.7) [textColor] {\footnotesize$f'(x)>0$};

          %% \node at (axis cs:-2.5,-.5) [anchor=north,textColor] {\footnotesize Decreasing};
          %% \node at (axis cs:.5,-.5) [anchor=north,textColor] {\footnotesize Decreasing};
          %% \node at (axis cs:-1,-.5) [anchor=north,textColor] {\footnotesize Increasing};
          %% \node at (axis cs:2,-.5) [anchor=north,textColor] {\footnotesize Increasing};

        \end{axis}
\end{tikzpicture}
\end{image}

Hence $f$ is increasing on $(-2,0)$ and $(1,\infty)$ and $f$ is
decreasing on $(-\infty,-2)$ and $(0,1)$. Moreover, from the first
derivative test, the local maximum is at $x=0$ while the local minima
are at $x=-2$ and $x=1$, see the graphs of of $f(x) =x^4/4 + x^3/3
-x^2$ and $f'(x) = x^3 + x^2 -2x$.
\begin{image}
\begin{tikzpicture}
	\begin{axis}[
            domain=-4:4,
            ymax=5,
            ymin=-5,
            %samples=100,
            axis lines =middle, xlabel=$x$, ylabel=$y$,
            every axis y label/.style={at=(current axis.above origin),anchor=south},
            every axis x label/.style={at=(current axis.right of origin),anchor=west}
          ]
          \addplot [dashed, textColor, smooth] plot coordinates {(-2,0) (-2,-2.667)}; %% {.451};
          \addplot [dashed, textColor, smooth] plot coordinates {(1,0) (1,-.4167)}; %% axis{2.215};

          \addplot [very thick, penColor, smooth] {(x^4)/4 + (x^3)/3 -x^2};
          \addplot [very thick, penColor2, smooth] {x^3 + x^2 -2*x};

          \node at (axis cs:-1.3,-2) [anchor=west] {\color{penColor}$f$};  
          \node at (axis cs:-2.1,2) [anchor=west] {\color{penColor2}$f'$};

          \addplot[color=penColor2,fill=penColor2,only marks,mark=*] coordinates{(-2,0)};  %% closed hole
          \addplot[color=penColor2,fill=penColor2,only marks,mark=*] coordinates{(1,0)};  %% closed hole
          \addplot[color=penColor2,fill=penColor3,only marks,mark=*] coordinates{(0,0)};  %% closed hole
          \addplot[color=penColor,fill=penColor,only marks,mark=*] coordinates{(-2,.-2.667)};  %% closed hole
          \addplot[color=penColor,fill=penColor,only marks,mark=*] coordinates{(1,-.4167)};  %% closed hole
        \end{axis}
\end{tikzpicture}
\end{image}
\end{explanation}
\end{example}


Hence we have seen that if $f'$ is zero and increasing at a point,
then $f$ has a local minimum at the point. If $f'$ is zero and
decreasing at a point then $f$ has a local maximum at the
point. Thus, we see that we can gain information about $f$ by
studying how $f'$ changes. This leads us to our next section.




%%CONCAVITY%%

\section{Concavity}
\begin{abstract}
  Here we examine what the second derivative tells us about the
  geometry of functions.
\end{abstract}

We know that the sign of the derivative tells us whether a function is
increasing or decreasing at some point. Likewise, the sign of the
second derivative $f''(x)$ tells us whether $f'(x)$ is increasing or
decreasing at $x$. We summarize the consequences of this seemingly
simple idea in the table below:

\begin{image}
  \begin{tikzpicture}
    \draw (0,0) -- (0,12);
    \draw (0,0) -- (12,0);
    \draw (6,0) -- (6,12);
    \draw (0,6) -- (12,6);
    \draw (12,0) -- (12,12);
    \draw (0,12) -- (12,12);
    
    \node at (-1.3,9) {\Large$0<f''(x)$};
    \node at (-1.3,3) {\Large$f''(x)<0$};
    \node at (3,12.4) {\Large$f'(x)<0$};
    \node at (9,12.4) {\Large$0<f'(x)$};
    
    \draw [penColor,ultra thick,domain=180:270] plot ({2*cos(\x)+4}, {2*sin(\x)+11});
    \draw [penColor,ultra thick,domain=270:360] plot ({2*cos(\x)+8}, {2*sin(\x)+11});
    \draw [penColor,ultra thick,domain=0:90] plot ({2*cos(\x)+2}, {2*sin(\x)+3});
    \draw [penColor,ultra thick,domain=180:90] plot ({2*cos(\x)+10}, {2*sin(\x)+3});

    \node at (3,7.5) [text width=5cm] {\large
      Here $y=f(x)$ is decreasing, while the rate itself is increasing.
      In this case the curve is \dfn{concave up}.};

    \node at (9,7.5) [text width=5cm] {\large
      Here $y=f(x)$ is increasing, while the rate itself is increasing.
      In this case the curve is \dfn{concave up}.};

    \node at (3,1.5) [text width=5cm] {\large
      Here $y=f(x)$ is decreasing, while the rate itself is decreasing.
      In this case the curve is \dfn{concave down}.};

    \node at (9,1.5) [text width=5cm] {\large
      Here $y=f(x)$ is increasing, while the rate itself is decreasing.
      In this case the curve is \dfn{concave down}.};
  \end{tikzpicture}
\end{image}

If we are trying to understand the shape of the graph of a function,
knowing where it is concave up and concave down helps us to get a more
accurate picture. It is worth summarizing what we have seen already in
to a single theorem.

\begin{theorem}[Test for Concavity]\index{concavity test}
Suppose that $f''(x)$ exists on an interval.
\begin{enumerate}
\item $f''(x)>0$ on that interval whenever $y=f(x)$ is concave up on that interval.
\item $f''(x)<0$ on that interval whenever $y=f(x)$ is concave down on that interval.
\end{enumerate}
\end{theorem}


\begin{example}
  Let $f$ be a continuous function and suppose that:
  \begin{itemize}
  \item $f'(x) > 0$ for $-1< x<1$.
  \item $f'(x) < 0$ for $-2< x<-1$ and $1<x<2$.
  \item $f''(x) > 0$ for $-2<x<0$ and $1<x< 2$.
  \item $f''(x) < 0$ for $0<x< 1$.  
  \end{itemize}
  Sketch a possible graph of $f$.
  \begin{explanation}
    Start by marking where the derivative changes sign and indicate
    intervals where $f$ is increasing and intervals $f$ is
    decreasing. The function $f$ has a negative derivative from $-2$
    to $x=\answer[given]{-1}$. This means that $f$ is
    \wordChoice{\choice{increasing}\choice[correct]{decreasing}} on
    this interval. The function $f$ has a positive derivative from
    $x=\answer[given]{-1}$ to $x=\answer[given]{1}$. This means that
    $f$ is
    \wordChoice{\choice[correct]{increasing}\choice{decreasing}} on
    this interval. Finally, The function $f$ has a negative derivative
    from $x=\answer[given]{1}$ to $2$. This means that $f$ is
    \wordChoice{\choice{increasing}\choice[correct]{decreasing}} on
    this interval.
  \begin{image}
    \begin{tikzpicture}
    \begin{axis}[
        xmin=-2,xmax=2,ymin=-2,ymax=2,
        axis lines=center,
        width=6in,
        height=3in,
        every axis y label/.style={at=(current axis.above origin),anchor=south},
        every axis x label/.style={at=(current axis.right of origin),anchor=west},
      ]
      \addplot [dashed, penColor2] plot coordinates {(-1,-2) (-1,2)}; %% Critical points
      \addplot [dashed, penColor2] plot coordinates {(1,-2) (1,2)}; %% Critical points

      \addplot [->, line width=10, penColor!10!background] plot coordinates {(-1+.2,-2+.2) (1-.2,2-.2)};
      \addplot [->, line width=10, penColor!10!background] plot coordinates {(-2+.2,2-.2) (-1-.2,-2+.2)};
      \addplot [->, line width=10, penColor!10!background] plot coordinates {(1+.2,2-.2) (2-.2,-2+.2)}; 
      
      %\addplot [very thick,penColor,smooth, domain=(-2:2)] {x^3+x^2-2*x)};
    \end{axis}
  \end{tikzpicture}
  \end{image}
  Now we should sketch the concavity: \wordChoice{\choice[correct]{concave up}\choice{concave down}} when the second
  derivative is positive, \wordChoice{\choice{concave up}\choice[correct]{concave down}} when the second derivative is
  negative.
    \begin{image}
    \begin{tikzpicture}
    \begin{axis}[
        xmin=-2,xmax=2,ymin=-2,ymax=2,
        axis lines=center,
        width=6in,
        height=3in,
        every axis y label/.style={at=(current axis.above origin),anchor=south},
        every axis x label/.style={at=(current axis.right of origin),anchor=west},
      ]
      \addplot [dashed, penColor2] plot coordinates {(-1,-2) (-1,2)}; %% Critical points
      \addplot [dashed, penColor2] plot coordinates {(1,-2) (1,2)}; %% Critical points

      %\addplot [->, line width=10, penColor!10!background] plot coordinates {(-1+.2,-2+.2) (1-.2,2-.2)};
      %\addplot [->, line width=10, penColor!10!background] plot coordinates {(-2+.2,2-.2) (-1-.2,-2+.2)};
      %\addplot [->, line width=10, penColor!10!background] plot coordinates {(1+.2,2-.2) (2-.2,-2+.2)};

      \addplot [penColor3!20!background,line width=10,domain=180:270] ({-1.1+.7*cos(x)}, {1.2+2*sin(x)});
      \addplot [penColor3!20!background,line width=10,domain=270:360] ({-.9+.7*cos(x)}, {-.1+.7*sin(x)});
      \addplot [penColor3!20!background,line width=10,domain=180:90] ({.9+.7*cos(x)}, {.1+.7*sin(x)});
      \addplot [penColor3!20!background,line width=10,domain=180:265] ({1.9+.7*cos(x)}, {.9+ 2*sin(x)});

      \addplot [->, line width=10, penColor3!20!background] plot coordinates {(-1.1,-.1-.7) (-1,-.1-.7)};
      \addplot [->, line width=10, penColor3!20!background] plot coordinates {(.9,.1+.7) (1,.1+.7)};
      
      \addplot [->, line width=10, penColor3!20!background] plot coordinates {(-.9+.7,-.2) (-.9+.8,.2)};
      \addplot [->, line width=10, penColor3!20!background] plot coordinates {(1.8,-1.1) (2,-1.2)};
      
      %\addplot [very thick,penColor,smooth, domain=(-2:2)] {x^3+x^2-2*x)};
    \end{axis}
  \end{tikzpicture}
    \end{image}
    Finally, we can sketch our curve:
        \begin{image}
    \begin{tikzpicture}
    \begin{axis}[
        xmin=-2,xmax=2,ymin=-2,ymax=2,
        axis lines=center,
        width=6in,
        height=3in,
        every axis y label/.style={at=(current axis.above origin),anchor=south},
        every axis x label/.style={at=(current axis.right of origin),anchor=west},
      ]
      \addplot [dashed, penColor2] plot coordinates {(-1,-2) (-1,2)}; %% Critical points
      \addplot [dashed, penColor2] plot coordinates {(1,-2) (1,2)}; %% Critical points

      \addplot [penColor,ultra thick,domain=-2:1,smooth] {(-x^3+3*x)*.5};
      \addplot [penColor,ultra thick,domain=1:2,smooth] {(-(x-3)^3+3*(x-3))*.5};
    \end{axis}
  \end{tikzpicture}
  \end{image}
  \end{explanation}
\end{example}


%%INFLECTION POINTS%%
\section{Inflection points}


If we are trying to understand the shape of the graph of a function,
knowing where it is concave up and concave down helps us to get a more
accurate picture. It is worth summarizing what we have seen already in
to a single theorem.

\begin{theorem}[Test for Concavity]\index{concavity test}
Suppose that $f''(x)$ exists on an interval.
\begin{enumerate}
\item If $f''(x)>0$ on an interval, then $f$ is concave up on that interval.
\item If $f''(x)<0$ on an interval, then $f$ is concave down on that interval.
\end{enumerate}
\end{theorem}


Of particular interest are points at which the concavity changes from
up to down or down to up. 

\begin{definition}\index{inflection point}
If $f$ is continuous and its concavity changes either from up to down
or down to up at $x=a$, then $f$ has an \dfn{inflection point} at
$x=a$.
\end{definition}

It is instructive to see some examples of inflection points:
\begin{image}
\begin{tikzpicture}
	\begin{axis}[
            %height=7cm,
            %width=2in,
            width=6in,
            height=2in,
            %ymax=8,
            %ymin=-1,
            axis lines=none,
            clip=false,
          ]
          \addplot [very thick, penColor, smooth, domain=(0:1)] {(x-1)^2+1};
          \addplot [very thick, penColor, smooth, domain=(1:2)] {-(x-1)^2+1};
          \addplot[color=penColor,fill=penColor,only marks,mark=*] coordinates{(1,1)};
          \node at (axis cs:1,-.5) [text width=2in] {This is an inflection point. The concavity changes from concave up to concave down.};
        
          \addplot [very thick, penColor, smooth,domain=(4:5)] {-sqrt(abs(1-(x-4)))+1};
          \addplot [very thick, penColor, smooth,domain=(5:6)] {sqrt((x-4)-1)+1};
          \addplot[color=penColor,fill=penColor,only marks,mark=*] coordinates{(5,1)};
          \node at (axis cs:5,-.5) [text width=2in] {This is an inflection point. The concavity changes from concave up to concave down.};
        \end{axis}
\end{tikzpicture}
\end{image}

It is also instructive to see some nonexamples of inflection points:
\begin{image}
\begin{tikzpicture}
	\begin{axis}[
            %height=7cm,
            %width=2in,
            width=6in,
            height=2in,
            %ymax=8,
            %ymin=-1,
            axis lines=none,
            clip=false,
          ]
          \addplot [very thick, penColor2, smooth, domain=(0:2)] {-(x-1)^2+1};
          \addplot[color=penColor2,fill=penColor2,only marks,mark=*] coordinates{(1,1)};
          \node at (axis cs:1,-.5) [text width=2in] {This is \textbf{not} an inflection point. The curve is concave down on either side of the point.};
          \addplot [very thick, penColor2, smooth,domain=(4:5)] {sqrt(abs(1-(x-4)))};
          \addplot [very thick, penColor2, smooth,domain=(5:6)] {sqrt(x-5)};
          \addplot[color=penColor2,fill=penColor2,only marks,mark=*] coordinates{(5,0)};
          \node at (axis cs:5,-.5) [text width=2in] {This is \textbf{not} an inflection point. The curve is concave down on either side of the point.};
        \end{axis}
\end{tikzpicture}
\end{image}

We identify inflection points by first finding $x$ such that $f''(x)$
is zero or undefined and then checking to see whether $f''(x)$ does in
fact go from positive to negative or negative to positive at these
points.

\begin{warning}
Even if $f''(a) = 0$, the point determined by $x=a$ might \textbf{not}
be an inflection point.
\end{warning}


\begin{example}
Describe the concavity of $f(x)=x^3-x$. 

\begin{explanation}
To start, compute the first and second derivative of $f(x)$ with
respect to $x$,
\[
f'(x)=\answer[given]{3x^2-1}\qquad\text{and}\qquad f''(x)=\answer[given]{6x}.
\]
Since $f''(0)=0$, there is potentially an inflection point at
$x=0$. Using test points, we note the concavity does change from down
to up, hence there is an inflection point at $x=0$. The curve is
concave down for all $x<0$ and concave up for all $x>0$, see the
graphs of $f(x) = x^3-x$ and $f''(x) = 6x$.
\begin{image}
\begin{tikzpicture}
	\begin{axis}[
            domain=-3:3,
            ymax=3,
            ymin=-3,
            axis lines =middle, xlabel=$x$, ylabel=$y$,
            every axis y label/.style={at=(current axis.above origin),anchor=south},
            every axis x label/.style={at=(current axis.right of origin),anchor=west}
          ]
          \addplot [very thick, penColor, smooth] {x^3-x};
          \addplot [very thick, penColor4, smooth] {6*x};         
          \node at (axis cs:-.75,.6) [anchor=west] {\color{penColor}$f$};  
          \node at (axis cs:.2,1) [anchor=west] {\color{penColor4}$f''$};
          \addplot[color=penColor4!50!penColor,fill=penColor4!50!penColor,only marks,mark=*] coordinates{(0,0)};  %% closed hole
        \end{axis}
\end{tikzpicture}
%% \caption{A plot of $f(x) = x^3-x$ and $f''(x) = 6x$. We can see that
%%   the concavity change at $x=0$.}
%% \label{figure:3x^2-1}
%% \end{marginfigure}
\end{image}
\end{explanation}
\end{example}


Note that we need to compute and analyze the second derivative to
understand concavity, so we may as well try to use the second
derivative test for maxima and minima. If for some reason this fails
we can then try one of the other tests.

\section{The second derivative test}


Recall the first derivative test:
\begin{itemize}
\item If $f'(x)>0$ to the left of $a$ and $f'(x)<0$ to the right of
  $a$, then $f(a)$ is a local maximum.
\item If $f'(x)<0$ to the left of $a$ and $f'(x)>0$ to the right of
  $a$, then $f(a)$ is a local minimum.
\end{itemize}

If $f'$ changes from positive to negative it is decreasing. In this
case, $f''$ might be negative, and if in fact $f''$ is negative
then $f'$ is definitely decreasing, so there is a local maximum at
the point in question. On the other hand, if $f'$ changes from
negative to positive it is increasing. Again, this means that
$f''$ might be positive, and if in fact $f''$ is positive then
$f'$ is definitely increasing, so there is a local minimum at the
point in question. We summarize this as the \textit{second derivative
  test}.

\begin{theorem}[Second Derivative Test]\index{second derivative test}\label{T:sdt}
Suppose that $f''(x)$ is continuous on an open interval and that
$f'(a)=0$ for some value of $a$ in that interval.
\begin{itemize}
\item If $f''(a) <0$, then $f$ has a local maximum at $a$.
\item If $f''(a) >0$, then $f$ has a local minimum at $a$.
\item If $f''(a) =0$, then the test is inconclusive. In this case,
  $f$ may or may not have a local extremum at $x=a$.
\end{itemize}
\end{theorem}


The second derivative test is often the easiest way to identify local
maximum and minimum points. Sometimes the test fails and sometimes
the second derivative is quite difficult to evaluate. In such cases we
must fall back on one of the previous tests.

\begin{example}
Once again, consider the function 
\[
f(x) = \frac{x^4}{4}+\frac{x^3}{3}-x^2
\]
Use the second derivative test, to locate the
local extrema of $f$.

\begin{explanation}
Start by computing
\[
f'(x) = \answer[given]{x^3 + x^2 -2x} \qquad\text{and}\qquad f''(x) = \answer[given]{3x^2 + 2x-2}.
\] 
Using the same technique as we used before, we find that 
\[
f'(-2) = \answer[given]{0},\qquad f'(0) = \answer[given]{0}, \qquad f'(1) = \answer[given]{0}. 
\]
Now we'll attempt to use the second derivative test,
\[
f''(-2) = \answer[given]{6}, \qquad f''(0) =\answer[given]{ -2}, \qquad f''(1) = \answer[given]{3}.
\]
Hence we see that $f$ has a local minimum at $x=-2$, a local
maximum at $x=0$, and a local minimum at $x=1$, see below for a plot
of $f(x) =x^4/4 + x^3/3 -x^2$ and $f''(x) = 3x^2 + 2x -2$:
\begin{image}
\begin{tikzpicture}
	\begin{axis}[
            domain=-4:4,
            ymax=7,
            ymin=-4,
            %samples=100,
            axis lines =middle, xlabel=$x$, ylabel=$y$,
            every axis y label/.style={at=(current axis.above origin),anchor=south},
            every axis x label/.style={at=(current axis.right of origin),anchor=west}
          ]
          \addplot [dashed, textColor, smooth] plot coordinates {(-2,-2.667) (-2,6)}; %% {.451};
          \addplot [dashed, textColor, smooth] plot coordinates {(1,0) (1,3)}; %% axis{2.215};

          \addplot [very thick, penColor, smooth] {(x^4)/4 + (x^3)/3 -x^2};
          \addplot [very thick, penColor4, smooth] {3*x^2 + 2*x -2};

          \node at (axis cs:-1.7,-2.7) [anchor=west] {\color{penColor}$f$};  
          \node at (axis cs:-1.5,2) [anchor=west] {\color{penColor4}$f''$};

          \addplot[color=penColor4,fill=penColor4,only marks,mark=*] coordinates{(-2,6)};  %% closed hole
          \addplot[color=penColor4,fill=penColor4,only marks,mark=*] coordinates{(1,3)};  %% closed hole
          \addplot[color=penColor4,fill=penColor4,only marks,mark=*] coordinates{(0,-2)};  %% closed hole
          \addplot[color=penColor,fill=penColor,only marks,mark=*] coordinates{(0,0)};  %% closed hole
          \addplot[color=penColor,fill=penColor,only marks,mark=*] coordinates{(-2,.-2.667)};  %% closed hole
          \addplot[color=penColor,fill=penColor,only marks,mark=*] coordinates{(1,-.4167)};  %% closed hole
        \end{axis}
\end{tikzpicture}
%% \caption{A plot of $f(x) =x^4/4 + x^3/3 -x^2$ and $f''(x) = 3x^2 + 2x -2$.}
%% \label{figure:SDT(x^4)/4 + (x^3)/3 -x^2}
\end{image}

\end{explanation}
\end{example}


\begin{problem}
  If $f''(a)=0$, what does the second derivative test tell us?
  \begin{multipleChoice}
    \choice{The function has a local extrema at $x=a$.}
    \choice{The function does not have a local extrema at $x=a$.}
    \choice[correct]{It gives no information on whether $x=a$ is a local extremum.} 
  \end{multipleChoice}
  
\end{problem}


\end{document}
