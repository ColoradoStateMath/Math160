\documentclass[handout]{ximera}
%\documentclass[10pt,handout,twocolumn,twoside,wordchoicegiven]{xercises}
%\documentclass[10pt,handout,twocolumn,twoside,wordchoicegiven]{xourse}

%\author{Steven Gubkin}
%\license{Creative Commons 3.0 By-NC}
\input{../preamble.tex}

\outcome{Practice Antiderivatives Concepts.}
   

\title{Antiderivatives: True/False}

\begin{document}
\begin{abstract}
  Here we'll practice T/F with antiderivatives.
\end{abstract}
\maketitle

\begin{exercise}
If $f$ and $g$ both have the same antiderivative, then $f = g$	

	\begin{multipleChoice}	
		\choice[correct]{True}
		\choice{False}
	\end{multipleChoice}

\end{exercise}

\begin{exercise}
Let $F$ be an antiderivative of $f$, and $G$ be an antiderivative of $g$.  If $f(x)$ is always less than $g(x)$, then $F(x)$ is always less than $G(x)$

	\begin{multipleChoice}	
		\choice{True}
		\choice[correct]{False}
	\end{multipleChoice}

\end{exercise}


\begin{exercise}
Let $F$ be an antiderivative of $f$.  If $f$ is increasing on an interval, then $F$ must be positive on that interval.

	\begin{multipleChoice}	
		\choice{True}
		\choice[correct]{False}
	\end{multipleChoice}

\end{exercise}


\begin{exercise}
Let $F$ and $G$ be antiderivatives of $f$, which is continuous on the whole real line.   Then $F(1)-F(0) = G(1)-G(0)$

	\begin{multipleChoice}	
		\choice[correct]{True}
		\choice{False}
	\end{multipleChoice}

\end{exercise}

\begin{exercise}
If $f'(x) = x^2$, then $f$ is an antiderivative of $x^2$.

	\begin{multipleChoice}	
		\choice[correct]{True}
		\choice{False}
	\end{multipleChoice}

\end{exercise}

\begin{exercise}
The antiderivative of a constant function is a linear function.

	\begin{multipleChoice}	
		\choice[correct]{True}
		\choice{False}
	\end{multipleChoice}

\end{exercise}

\begin{exercise}
If $F$ is an antiderivative of $f$, which is defined on the whole real line, then $F$ is continuous.

	\begin{multipleChoice}	
		\choice[correct]{True}
		\choice{False}
	\end{multipleChoice}

\end{exercise}























\end{document}
