\documentclass[handout]{ximera}
%\documentclass[10pt,handout,twocolumn,twoside,wordchoicegiven]{xercises}
%\documentclass[10pt,handout,twocolumn,twoside,wordchoicegiven]{xourse}

%\author{Steven Gubkin}
%\license{Creative Commons 3.0 By-NC}
\input{../preamble.tex}

\outcome{Practice on area between curves.}
   

\title[Exercises:]{Area Between Curves Exercises}

\begin{document}
\begin{abstract}
  Here we'll practice finding area between curves.
\end{abstract}
\maketitle

%%Problem 1
\begin{exercise}
Find the area of the region bounded by the curves $y = 1$ and $y =7+6x-x^2$.

\begin{hint}
	First we find the points of intersections.
	
	\begin{align*}
	7+6x-x^2 &= 1\\
	x^2-6x-6 &=0\\
	(x-3)^2 - 15 &=0\\
    (x-3)^2 &= 15 \\
    x-3 &= \pm \sqrt(15) \\
	x &= 3 +\sqrt{15} \textrm{ or  }3 - \sqrt{15}
	\end{align*}
\end{hint}

\begin{hint}
	By graphing the two functions, we can see that $y =7+6x-x^2$ is always greater than $1$ on the interval $[3 - \sqrt{15}, 3 + \sqrt{15}]$.
\end{hint}

\begin{hint}
	Thus the area of this region is
	\[
	\int_{3-\sqrt{15}}^{3+\sqrt{15}} (7+6x-x^2) - 1 \d x
	\]
\end{hint}

\begin{hint}
	\begin{align*}
		\int_{3 - \sqrt{15}}^{3+\sqrt{15}} (7+6x-x^2) - 1 \d x &= \int_{3 - \sqrt{15}}^{3+\sqrt{15}} 6+6x-x^2 \d x\\
		&= \eval{6x+3x^2-\frac{1}{3}x^3}_{3 - \sqrt{15}}^{3+\sqrt{15}}\\
		&= 20\sqrt{15}
	\end{align*}
\end{hint}

	\[
		\textrm{Area} = \answer[tolerance = 0.0001]{20*(15)^{0.5}}
	\]

\end{exercise}

%%Problem 2
\begin{exercise}
Find the area of the region bounded by the curves $y = x$ and $y = \frac{1}{2} x + 1$ in the first quadrant.

\begin{hint}
	First we find the point of intersectios.
	
	\begin{align*}
	x &= \frac{1}{2}x+1\\
	\frac{1}{2}x &=1\\
	x &=2
	\end{align*}
\end{hint}

\begin{hint}
	$y = \frac{1}{2}x+1$ is greater than $y=x$ on the interval $[0,2]$
\end{hint}

\begin{hint}
	Thus the area of this region is

	\[
	\int_0^2 (\frac{1}{2}x+1) - x \d x
	\]
\end{hint}

\begin{hint}
	\begin{align*}
		\int_0^2 (\frac{1}{2}x+1) - x \d x &= \int_0^2 (\frac-{1}{2}x+1)  \d x\\
		&= \eval{-\frac{1}{4}x^2+x}_0^2\\
		&= 1
	\end{align*}
\end{hint}

	\[
		\textrm{Area} = \answer{1}
	\]
	
	\begin{feedback}
		Since this region is a triangle with base of length $1$ on the $y$-axis, and height $2$, one could also have calculated this area geometrically.
	\end{feedback}
\end{exercise}

%%Problem 3
\begin{exercise}
Find the area of the region bounded by two consecutive intersections of the curves $y=\sin(x)$ and $y = \cos(x)$.

\begin{hint}
  We can use any two consecutive intersections, but the first two positive intersections are convenient.

  These occur at $x = \frac{\pi}{4}$ and $x = \frac{3\pi}{4}$
\end{hint}

\begin{hint}
  $y = \sin(x)$ is greater than $y=\cos(x)$ on the interval $[\frac{\pi}{4},\frac{3\pi}{4}]$
\end{hint}

\begin{hint}
	Thus the area of this region is
	\[
	\int_{\frac{\pi}{4}}^{\frac{3\pi}{4}} \sin(x) - \cos(x) \d x
	\]
\end{hint}

\begin{hint}
  \begin{align*}
	  \int_{\frac{\pi}{4}}^{\frac{3\pi}{4}} \sin(x) - \cos(x) \d x &=  -\eval{\cos(x)+\sin(x)}_{\frac{\pi}{4}}^{\frac{3\pi}{4}}  \d x\\
		&=-(-\frac{2}{\sqrt{2}} - \frac{2}{\sqrt{2}})\\
		&= \frac{4}{\sqrt{2}}
	\end{align*}
\end{hint}

  \[
  \textrm{Area} = \answer{\frac{4}{\sqrt{2}}}
  \]

\end{exercise}

%%Problem 4
\begin{exercise}
Find the area of the region bounded by the vertical lines $x=0$ and $x=1$, and the curves $y = x^2+1$ and $y = x^3$.

\begin{hint}
	There are no intersection points between the curves on this interval, and $y = x^2+1$ is always above $y=x^3$ on this interval.
\end{hint}


\begin{hint}
	Thus the area of this region is

	\[
	\int_{0}^{1} (x^2+1)-x^3 \d x
	\]
\end{hint}

\begin{hint}
	\begin{align*}
		\int_{0}^{1} (x^2+1)-x^3 \d x &=  \eval{\frac{x^3}{3}+x-\frac{x^4}{4}}_0^1  \d x\\
		&= \frac{13}{12}
	\end{align*}
\end{hint}

	\[
		\textrm{Area} = \answer{\frac{13}{12}}
	\]

\end{exercise}

%%Problem 5
\begin{exercise}
Find the area of the region bounded by the curves $y=x-2$ and $y=-x$, and $y = 4+2x-x^2$.

\begin{hint}

	First sketch a picture of the region

\begin{image}
\begin{tikzpicture}
	\begin{axis}[
            domain=-2:5, ymax=5,xmax=5,ymin=-2, xmin=-2,
            axis lines =center, xlabel=$x$, ylabel=$y$,
 	   xtick={-1,1,3},
            xticklabels={$a$,$b$,$c$}, 
            every axis y label/.style={at=(current axis.above origin),anchor=south},
            every axis x label/.style={at=(current axis.right of origin),anchor=west},
            axis on top,
          ]
          \addplot [draw=none,fill=fillp,domain=-1:3] {4+2*x-x^2} \closedcycle;
          \addplot [draw=none,fill=background,domain=-1:0] {-x} \closedcycle;
          \addplot [draw=none,fill=background,domain=2:3] {x-2} \closedcycle;
          \addplot [draw=none,fill=fillp,domain=0:1] {-x} \closedcycle;
          \addplot [draw=none,fill=fillp,domain=1:2] {x-2} \closedcycle;
          \addplot [draw=penColor,very thick,smooth] {-x};
          \addplot [draw=penColor2,very thick,smooth] {x-2};
          \addplot [draw=penColor3,very thick,smooth] {4+2*x-x^2};
         
        \end{axis}
\end{tikzpicture}
\end{image}
\end{hint}

\begin{hint}
	Next we find the $x$-coordinates of the relevant points of intersection.
	
	Let $a$ be the point of intersection between $y = -x$ and $y = 4+2x-x^2$, $b$ between $y=-x$ and $y=x-2$, and $c$ between $y=x-2$ and $y=4+2x-x^2$.

	Then
	
\[	
	\begin{cases}
		-a = 4+2a-a^2\\
		-b = b-2\\
		c-2 = 4+2c-c^2
	\end{cases}	
\]

\[	
	\begin{cases}
		0 = 4+3a-a^2\\
		 0 = 2b-2\\
		0= 6+c-c^2
	\end{cases}	
\]

\[	
	\begin{cases}
		0 = (1+a)(4-a)\\
		 b = 1\\
		0= (3-c)(2+c)
	\end{cases}	
\]

Thus $a=-1$,  $b = 1$, and $c = 3$ (make sure you understand why $a=4$ and $c=-2$ are not the correct intersections).
\end{hint}


\begin{hint}
	Thus the area of this region is

	\[
	\int_{-1}^{1} (4+2x-x^2) - (-x) \d x + \int_{1}^{3} (4+2x-x^2) - (x-2) \d x
	\]
\end{hint}

\begin{hint}
	The first summand is

	\begin{align*}
		\int_{-1}^{1} (4+2x-x^2) - (-x) \d x &=  \int_{-1}^{1} (4+3x-x^2) \d x\\
		&= \eval{4x+\frac{3}{2}x^2-\frac{1}{3}x^3}_{-1}^1\\
		&= \frac{22}{3}
	\end{align*}

	The second summand is

	\begin{align*}
		\int_{1}^{3} (4+2x-x^2) - (x-2) \d x &= \int_{1}^{3} (6+x-x^2) \d x\\
		&= \eval{6x+\frac{x^2}{2}-\frac{x^3}{3}}_1^3\\
		&= \frac{22}{3}
	\end{align*}

	Note that these two area equal, which we could have also discovered by observing the symmetry of the figure about the line $x=1$.
\end{hint}

	\[
		\textrm{Area} = \answer{\frac{44}{3}}
	\]

\end{exercise}

%%Problem 6
\begin{exercise}
Find the area of the region bounded by the curves $y=\sqrt[3]{x}$ and $y=\frac{2}{7}(x-1)$ and the $x$-axis.  Try to solve this problem using horizontal rectangles, i.e. by integration with respect to $y$.

\begin{hint}

	First sketch a picture of the region

\begin{image}
\begin{tikzpicture}
	\begin{axis}[
            domain=0:10, ymax=3,xmax=10,ymin=0, xmin=0,
            axis lines =center, xlabel=$x$, ylabel=$y$,
 	   xtick={1,8},
            xticklabels={$1$,$8$}, 
            every axis y label/.style={at=(current axis.above origin),anchor=south},
            every axis x label/.style={at=(current axis.right of origin),anchor=west},
            axis on top,
          ]
         
         \addplot [draw=none,fill=fillp,domain=0:8,samples=200] {x^(1/3)} \closedcycle;
         \addplot [draw=none,fill=background,domain=1:8] {2*(x-1)/7} \closedcycle;
          \addplot [draw=penColor,very thick,smooth] {2*(x-1)/7};
          \addplot [draw=penColor2,very thick,smooth, samples=200] {x^(1/3)};
	 \addplot [textColor,dashed] plot coordinates {(8,0) (8,5)};
	 \addplot [textColor,dashed] plot coordinates {(0,2) (10,2)};
        \end{axis}
\end{tikzpicture}
\end{image}
\end{hint}

\begin{hint}
Rather than using two different regions of integration, this problem is a great case opportunity to use horizontal rectangles.

Solving for $x$, we can rewrite the curves as $x = y^3$ and $x = \frac{7y}{2}+1$.

The curves intersect at the point $(8,2)$
\end{hint}

\begin{hint}
	The area of this region is

	\[
	\int_{0}^{2} ( \frac{7y}{2}+1)-y^3 \d y
	\]
\end{hint}

\begin{hint}


	\begin{align*}
		\int_{0}^{2} ( \frac{7y}{2}+1)-y^3 \d y &=  \eval{\frac{7y^2}{4}+y-\frac{y^4}{4}}_0^2\\
		&= 7+2-4\\
		&= 5
	\end{align*}

	
\end{hint}

	\[
		\textrm{Area} = \answer{5}
	\]

\end{exercise}











\end{document}
