\documentclass[handout]{ximera}
%\documentclass[10pt,handout,twocolumn,twoside,wordchoicegiven]{xercises}
%\documentclass[10pt,handout,twocolumn,twoside,wordchoicegiven]{xourse}

%\author{Steven Gubkin}
%\license{Creative Commons 3.0 By-NC}
\input{../preamble.tex}

\outcome{Practice Derivatives.}
   

\title{Derivative Exercises}

\begin{document}
\begin{abstract}
  Here we'll practice derivative rules.
\end{abstract}
\maketitle

%%basic Deriv rules 3
\begin{exercise}
Let $Y(z) = -2 z^2 + \sqrt[5]{z} -3 \sin (z)$. Compute:

\begin{hint}
Rewrite $\sqrt[5]{z}$ as $z^{1/5}$.
\end{hint}

\[
Y'(z)
=\answer{-4z+\frac{1}{5 (z)^{4/5}}-3 \cos(z)}
\]
\end{exercise}


%%basic Deriv rules by Mary
\begin{exercise}
Let $f(x) = 3x^5+\frac{1}{\sqrt[3]{x}-3 \cos (x)}$. Compute:

\begin{hint}
Rewrite $\frac{1}{\sqrt[3]{x}}$ as $x^{-1/3}$.
\end{hint}
\[
f'(x)
=\answer[given]{15x^4-\frac{1/(3*x^{2/3}) + 3\sin(x)}{(x^{1/3} - 3\cos(x))^2}}
\]
\end{exercise}

%%basic product rule 1
\begin{exercise}
Let $R(v) = \left(\sqrt{v}-\frac{2}{\sqrt[4]{v}}\right) \left(3 \sin (v)-\frac{4}{v^5}\right)$. Compute:
\[
\dd{v}R(v)
=\answer[given]{\left(\sqrt{v}-\frac{2}{{v}^{1/4}}\right) \left(\frac{20}{v^6}+3 \cos (v)\right) + \left(\frac{1}{2 {v}^{5/4}}+\frac{1}{2 \sqrt{v}}\right) \left(3 \sin (v)-\frac{4}{v^5}\right)}
\]
\end{exercise}

%%basic product rule 3
\begin{exercise}
Let $p(n) = \left(-2 \sqrt[5]{n}-\frac{4}{\sqrt[4]{n}}\right) \left(3 n^4-2 \sin (n)\right)$. Compute:
\[
p'(n)
=\answer[given]{\left(\frac{1}{{n}^{5/4}}-\frac{2}{5 {n}^{4/5}}\right) \left(3 n^4 - 2 \sin (n)\right)+\left(-2 {n}^{1/5}-\frac{4}{{n}^{1/4}}\right) \left(12 n^3 - 2 \cos (n)\right)}
\]
\end{exercise}

%%basic product rule 5
\begin{exercise}
Let $R(u) = \frac{5 u^4+3 {u}^{1/3}}{-5 {u}^{1/3}}$. Compute:
\[
R'(u)
=\answer{\frac{5 u^4 + 3 {u}^{1/3}}{15 u {u}^{1/3}}-\frac{20 u^3+\frac{1}{{u}^{2/3}}}{5 {u}^{1/3}}}
\]
\end{exercise}

%%basic quotient rule 7
\begin{exercise}
Let $C(t) = \frac{ \sqrt{t}-\frac{3}{\sqrt[4]{t}}}{-2 \sqrt{t}}$. Compute:
\[
C'(t)
=\answer[given]{\frac{-9}{8 {t}^{7/4}}}
\]
%%% This is actually correct
\end{exercise}

%%basic chain rule 1
\begin{exercise}
Let $r(z) = -\cot \left(2 \sqrt{2} {z}^{1/4}\right)$. Compute:

\[
\dd{z}r(z)
=\answer{\frac{\csc ^2\left(2 \sqrt{2} {z}^{1/4}\right)}{\sqrt{2} {z}^{3/4}}}
\]
\end{exercise}

%%basic chain rule 11
\begin{exercise}
Let $y(v) = 2 \sin (2 v)-4 \sin ^9(2 v)$. Compute:

\[
y'(v)
=\answer[given]{4 \cos (2 v)-72 (\sin (2 v))^8 \cos (2 v)}
\]
\end{exercise}


%%basic chain rule with table 3
\begin{exercise}
Let $Q(z) = \frac{A(z)}{B(z)}$

We have the following table of values for $A$ and $B$.

\[
\begin{array}{c|c|c|c|c}
 z & A(z) & A'(z) & B(z) & B'(z)\\ \hline
 1 & 5 & -9 & 4 & -7\\ 
 2 & -3 & 3 & -10 & 5\\
 3 & 2 &	6&	6	&-2\\
 4 & -7  &	10&	9	&-4\\
 5 & -1&	-7	&-1&	-8\\
\end{array}
\]




Use this data to compute $\eval{\frac{\d}{\d z} Q(z)}_{z = 2}$.

	$Q'(2) = \answer{-0.15}$

\end{exercise}


Let $a(x)=g(x)h(x)$ and $b(x) = x^2g(x)+(h(x))^2$ and $c(x)=g(h(x))$

We have the following table of values for $g$ and $h$.

\[
\begin{array}{c|c|c|c|c}
 x & g(x) & g'(x) & h(x) & h'(x)\\ \hline
 1 & 2 & 2 & 6 & -3\\ 
 2 & 5 & 2 & 5 & 2\\
 3 & 4 & 3 & 1 & -1\\ 
 4 & -1 & 0 & 2 & 5\\ 
 5 & 3 & 1 & 4 & 9\\ 
\end{array}
\]

%%Derivs Problem with table 1
\begin{exercise}
Use this data to compute $a'(3)$.

	$a'(3) = \answer{-1}$

\end{exercise}


%%Derivs Problem with table 2
\begin{exercise}
Use this data to compute $b'(2)$.

	$b'(2) = \answer{48}$

\end{exercise}


%%Derivs Problem with table 3
\begin{exercise}
Use this data to compute $c'(5)$.

	$c'(5) = \answer{0}$

\end{exercise}

%%Derivs with table 4
\begin{exercise}
Let $A(x) = \frac{\sqrt[3]{x}}{B(x)}$.  Assume that $B(8) = 3$ and $B'(3) = \frac{1}{8}$.  Calculate $A'(8)$.

	$A'(8) = \answer{0}$


\end{exercise}

%%ImpDerivs4
\begin{exercise}
Use implicit differentiation to find $\frac{dy}{dx}$ where
\[
x = y^2.
\]
\[
\frac{dy}{dx} = \answer{\frac{1}{2y}}
\]

\end{exercise}

%%ImpDerivs1
\begin{exercise}
Use implicit differentiation to find $\frac{dy}{dx}$ where
\[
y^2x = \sin(xy).
\]

\[
\frac{dy}{dx} = \answer{\frac{y(y-\cos(xy))}{x(-2y+\cos(xy))}}
\]

\end{exercise}

%%ImpDerivs5
\begin{exercise}
Use implicit differentiation to find $\frac{dy}{dx}$ where
\[
\sin(x+y)+y = 0.
\]
\begin{prompt}
\[
\frac{dy}{dx} = \answer{\frac{-\cos(x+y)}{1+\cos(x+y)}}
\]
\end{prompt}
\end{exercise}

%%ImpDerivs2
\begin{exercise}
Use implicit differentiation to find $\frac{dy}{dx}$ where
\[
y = \frac{x-2}{y+3}.
\]

\[
\frac{dy}{dx} = \answer{\frac{y+3}{7+x+6y+y^2}}
\]

\end{exercise}


%%Pick f, f', f'' 
\begin{exercise}
The image below shows the graph of $f$, $f'$, and $f''$.  Determine which graph corresponds to which function.

\begin{image}
\begin{tikzpicture}
	\begin{axis}[
            xmin=-2, xmax=3, ymin=-1,ymax=3,    
            unit vector ratio*=1 1 1,
            axis lines =middle, xlabel=$x$, ylabel=$y$,
            every axis y label/.style={at=(current axis.above origin),anchor=south},
            every axis x label/.style={at=(current axis.right of origin),anchor=west},
            ticks=none,
	   %xtick={-4,...,4}, ytick={-3,...,3},
            %xticklabels={-4,,-2,,0,,2,,4,,6}, yticklabels={,-2,,0,,2,,4,,6,,8,,10},
            %grid=major,width=4in,
            %grid style={dashed, gridColor},
          ]

          \addplot [very thick, penColor, samples=100,smooth, domain=(-6:6)] {x^2};
          \addplot [very thick, penColor2, samples=100,smooth, domain=(-6:6)] {2*x};
          \addplot [very thick, penColor3, samples=100,smooth, domain=(-6:6)] {2};

	\node at (axis cs: -1, 1 ) [penColor,anchor=east] {$B$};
          \node at (axis cs:-0.25, -0.5) [penColor2,anchor=east] {$A$};
          \node at (axis cs: 2, 1.8 ) [penColor3,anchor=west] {$C$};
        \end{axis}
\end{tikzpicture}
\end{image}


\begin{align*}
	f(x) &= \answer{B}\\
	f'(x) &= \answer{A}\\
	f''(x) &= \answer{C}\\
\end{align*}


\end{exercise}

%%Pick f, f', f'' 
\begin{exercise}
The image below shows the graph of $f$, $f'$, and $f''$.  Determine which graph corresponds to which function.

\begin{image}
\begin{tikzpicture}
	\begin{axis}[
            xmin=-1.7, xmax=3, ymin=-4,ymax=3,    
            unit vector ratio*=1 1 1,
            axis lines =middle, xlabel=$x$, ylabel=$y$,
            every axis y label/.style={at=(current axis.above origin),anchor=south},
            every axis x label/.style={at=(current axis.right of origin),anchor=west},
            ticks=none,
	   %xtick={-4,...,4}, ytick={-3,...,3},
            %xticklabels={-4,,-2,,0,,2,,4,,6}, yticklabels={,-2,,0,,2,,4,,6,,8,,10},
            %grid=major,width=4in,
            %grid style={dashed, gridColor},
          ]

          \addplot [very thick, penColor2, samples=100,smooth, domain=(-1.2:6)] {x*sin(deg(x))};
          \addplot [very thick, penColor3, samples=100,smooth, domain=(-1.2:6)] {x*cos(deg(x))+sin(deg(x)) };
          \addplot [very thick, penColor, samples=100,smooth, domain=(-1.2:6)] {-x*sin(deg(x))+2*cos(deg(x))};

	\node at (axis cs: -1.2, 1.11844690316) [penColor2,anchor=east] {$A$};
          \node at (axis cs:-1.2, -1.36686839134) [penColor3,anchor=east] {$C$};
          \node at (axis cs: -1.2, -0.3937313942 ) [penColor,anchor=east] {$B$};
        \end{axis}
\end{tikzpicture}
\end{image}


\begin{align*}
	f(x) &= \answer{A}\\
	f'(x) &= \answer{C}\\
	f''(x) &= \answer{B}\\
\end{align*}


\end{exercise}


%%Pick f, f', f'' 
\begin{exercise}
The image below shows the graph of $f$, $f'$, and $f''$.  Determine which graph corresponds to which function.

\begin{image}
\begin{tikzpicture}
	\begin{axis}[
            xmin=-2, xmax=2, ymin=-2.2,ymax=1.2,    
            unit vector ratio*=1 1 1,
            axis lines =middle, xlabel=$x$, ylabel=$y$,
            every axis y label/.style={at=(current axis.above origin),anchor=south},
            every axis x label/.style={at=(current axis.right of origin),anchor=west},
            ticks=none,
	   %xtick={-4,...,4}, ytick={-3,...,3},
            %xticklabels={-4,,-2,,0,,2,,4,,6}, yticklabels={,-2,,0,,2,,4,,6,,8,,10},
            %grid=major,width=5in,
            %grid style={dashed, gridColor},
          ]

          \addplot [very thick, penColor, samples=100,smooth, domain=(-2:2)] {1/(x^2+1)};
          \addplot [very thick, penColor2, samples=100,smooth, domain=(-2:2)] {-2*x/(x^2+1)^2};
          \addplot [very thick, penColor3, samples=100,smooth, domain=(-2:2)] {(-2+6*x^2)/(1+x^2)^3};

	\node at (axis cs: 0, 1.1) [penColor, anchor=west] {$A$};
          \node at (axis cs: 0, 0.1) [penColor2,anchor=west] {$B$};
          \node at (axis cs: 0, -2 ) [penColor3,anchor=west] {$C$};
        \end{axis}
\end{tikzpicture}
\end{image}


\begin{align*}
	f(x) &= \answer{A}\\
	f'(x) &= \answer{B}\\
	f''(x) &= \answer{C}\\
\end{align*}


\end{exercise}

%%Pick f, f', f'' 
\begin{exercise}
The image below shows the graph of $f$, $f'$, and $f''$.  Determine which graph corresponds to which function.

\begin{image}
\begin{tikzpicture}
	\begin{axis}[
            xmin=-1, xmax=0.8, ymin=-1.2,ymax=1.2,    
            unit vector ratio*=1 1 1,
            axis lines =middle, xlabel=$x$, ylabel=$y$,
            every axis y label/.style={at=(current axis.above origin),anchor=south},
            every axis x label/.style={at=(current axis.right of origin),anchor=west},
            ticks=none,
	   %xtick={-4,...,4}, ytick={-3,...,3},
            %xticklabels={-4,,-2,,0,,2,,4,,6}, yticklabels={,-2,,0,,2,,4,,6,,8,,10},
            %grid=major,width=4in,
            %grid style={dashed, gridColor},
          ]

          \addplot [very thick, penColor2, samples=100,smooth, domain=(-0.8:0.8)] {x^4+x^3-x};
          \addplot [very thick, penColor3, samples=100,smooth, domain=(-0.8:0.8)] {4*x^3+3*x^2-1 };
          \addplot [very thick, penColor, samples=100,smooth, domain=(-0.8:0.8)] {12*x^2+6*x};

	\node at (axis cs: -0.8, 0.71) [penColor2,anchor=east] {$B$};
          \node at (axis cs:-0.8, -1) [penColor3,anchor=east] {$C$};
          \node at (axis cs: -0.64, 1.05 ) [penColor,anchor=east] {$A$};
        \end{axis}
\end{tikzpicture}
\end{image}


\begin{align*}
	f(x) &= \answer{B}\\
	f'(x) &= \answer{C}\\
	f''(x) &= \answer{A}\\
\end{align*}

\end{exercise}


\end{document}
