\documentclass[handout]{ximera}
%\documentclass[10pt,handout,twocolumn,twoside,wordchoicegiven]{xercises}
%\documentclass[10pt,handout,twocolumn,twoside,wordchoicegiven]{xourse}

%\author{Steven Gubkin}
%\license{Creative Commons 3.0 By-NC}
\input{../preamble.tex}

\outcome{Practice Limits.}
   

\title[Exercises:]{Limits}

\begin{document}
\begin{abstract}
  Here we'll practice limits.
\end{abstract}
\maketitle

%%MADE UP BY MARY%%
\begin{exercise}
\[\lim_{x \to 5} \frac{x^2-25}{x-5} = \answer{10}\]
\begin{hint}
	\[x^2-25 = (x+5)(x-5)\]
\end{hint}
\end{exercise}


%%MADE UP BY MARY%%
\begin{exercise}
\[\lim_{x \to -4} \frac{x^2+3x-4}{x^2-4x-32} = \answer{5/13}\]
\begin{hint}
	\[x^2+3x-4 = (x+4)(x-1)\]
\end{hint}
\begin{hint}
	\[x^2-4x-32 = (x+4)(x-8)\]
\end{hint}
\end{exercise}

%%MADE UP BY MARY%%
\begin{exercise}
\[\lim_{x \to 0} \frac{x^2}{x^2-x} = \answer{0}\]
\begin{hint}
	\[x^2-x = x(x-1)\]
\end{hint}

\end{exercise}

%%MADE UP BY MARY%%
\begin{exercise}
\[\lim_{x \to 2} \frac{\frac{1}{2}-\frac{1}{x}}{2-x} = \answer{-1/4}\]
\begin{hint}
	\[\frac{1}{2}-\frac{1}{x} = \frac{x-2}{2x}\]
\end{hint}
\end{exercise}

%%MADE UP BY MARY%%
\begin{exercise}
\[\lim_{x \to 0} \frac{\tan(x)}{\sin(x)} = \answer{1}\]
\begin{hint}
	\[\tan(x) = \frac{\sin(x)}{\cos(x)}\]
\end{hint}
\end{exercise}

%%MADE UP BY MARY%%
\begin{exercise}
\[\lim_{x \to 0} \frac{\sqrt{9+x}-3}{x} = \answer{1/6}\]
\begin{hint}
	Multiply the numberator and denominator by $\sqrt{9+x}+3$.
\end{hint}
\end{exercise}

%% Question 1/2 from exercises
\begin{exercise}
\[\lim_{x \to 0} \frac{\sin(2x)}{3x} = \answer{2/3}\]
\begin{hint}
	\[\frac{\sin(2x)}{3x} = \frac{\sin(2x)}{2x} \frac{2}{3}\]
\end{hint}
\end{exercise}

%% Question 3 from exercises
\begin{exercise}
\[\lim_{x \to 0} \frac{\sin(5x)}{\sin(2x)} = \answer{5/2}\]
\begin{hint}
	\[ \frac{\sin(5x)}{\sin(2x)} = \frac{\sin(5x)}{5x} \frac{2x}{\sin(2x)} \frac{5}{2}\]
\end{hint}
\end{exercise}

%% Question 4 from exercises
\begin{exercise}
\[\lim_{x \to 0} \frac{\cos(x)-1}{x} = \answer{0}\]
\begin{hint}
	\[\frac{\cos(x)-1}{x} = \frac{(\cos(x)-1)(\cos(x)+1)}{x(\cos(x)+1)}\]
\end{hint}
\begin{hint}
	Expanding the numerator, and using the pythagorean identity should get you to the solution.
\end{hint}
\end{exercise}

%% Question 5 from exercises
\begin{exercise}
\[\lim_{x \to 0} \frac{\cos(2x)-1}{x\sin(3x)} = \answer{2/3}\]
\begin{hint}
	Multiple numerator and denominator by $\cos(2x)+1$, expand the numerator, use pythagorean identity, and should be almost home free.
\end{hint}
\end{exercise}

%% Question 6 from exercises
\begin{exercise}
	\[\lim_{h \to 0} \frac{\sqrt{2+h} - \sqrt{2}}{h} = \answer{1/(2\sqrt{2})}\]
	\begin{hint}
		Multiply numerator and denominator by $\sqrt{2+h}+\sqrt{2}$.
	\end{hint}
\end{exercise}

%% Question 8 from exercises
\begin{exercise}
If I want $\displaystyle\lim_{x \to 0} \frac{1-\cos(ax)}{x\sin(2x)} = 4$ what should I pick for $a$?
	\[a = \answer{8}\]
	\begin{hint}
		Multiply numerator and denominator by $1+\cos(ax)$, expand the numerator, and use the pythagorean identity.  This will let you find the limit in terms of $a$.  Then solve the equation to find the value of $a$.
	\end{hint}
\end{exercise}

%% Question 9 from exercises
\begin{exercise}
	\begin{warning}
		This problem is pretty hard.
	\end{warning}
	\[\lim_{h \to 0} \frac{\sin(1+h) - 2\sin(1)+\sin(1-h)}{h^2} = \answer{-\sin(1)}\]
	\begin{hint}
		Use the angle sum formula for sine on both $\sin(1+h)$ and $\sin(1-h)$
	\end{hint}
\end{exercise}

\begin{exercise}
\[
\lim_{\psi\to 4 } \frac{\sqrt{\psi +5}-3}{\psi -4}\begin{prompt} = \answer{\frac{1}{6}}\end{prompt}
\]
\begin{hint}
Multiply by $\frac{\sqrt{\psi +5}+3}{\sqrt{\psi +5}+3}$.
\end{hint}
\end{exercise}

%%piecewise limits 2%%
\begin{exercise}
\[
g(x) = \begin{cases}
  \frac{x^3 - 8}{x-2}  &\text{if $x<1$,} \\
  x^3+1 &\text{if  $x>1$.}
\end{cases}
\]
Does $\lim_{x \to 2} g(x)$ exist?  If it does, give its value.
Otherwise write DNE.
\begin{prompt}
\[
\lim_{x \to 2} g(x) = \answer{9}
\]
\end{prompt}
\begin{hint}
	Note that, close to $x=2$, the rule for $g(x)$ is $x^3+1$.
\end{hint}

\end{exercise}

%%limit law 3%%
\begin{exercise}
Suppose that $\displaystyle\lim_{u\to2}c(u)=-5$, $\displaystyle\lim_{u\to2}B(u)=-1$, and $\displaystyle\lim_{u\to2}Y(u)=-4$. Compute the limit

\[
\lim_{u\to 2 } \frac{c(u)}{B(u)-Y(u)}\begin{prompt} = \answer{-\frac{5}{3}}\end{prompt}
\]
\end{exercise}


%%Inf Lim 5%%
\begin{exercise}
Consider 
\[
f(\psi) = \frac{-5}{\psi -4}.
\]
Compute
\begin{enumerate}
\item $\displaystyle\lim_{\psi\to 4^- } f(\psi) \begin{prompt} = \answer{\infty}\end{prompt}$
\item $\displaystyle\lim_{\psi\to 4^+ } f(\psi) \begin{prompt} = \answer{-\infty}\end{prompt}$
\item $\displaystyle\lim_{\psi\to 4 } f(\psi) \begin{prompt} = \answer{DNE}\end{prompt}$
\end{enumerate}
\end{exercise}


%%Inf Lim sqrt 4%%
\begin{exercise}
Let 
\[
r(k) = \frac{\sqrt{k^6+5}-2 k^2}{k-k^3}.
\]
Compute
\begin{enumerate}
\item $\displaystyle\lim_{k\to \infty} r(k) \begin{prompt} = \answer{-1}\end{prompt}$
\item $\displaystyle\lim_{k\to -\infty}r(k) \begin{prompt} = \answer{1}\end{prompt}$
\end{enumerate}
\begin{hint}
Multiply by
\[
\frac{\frac{1}{k^3}}{\frac{1}{k^3}}
\]
\end{hint}
\end{exercise}




























\end{document}
