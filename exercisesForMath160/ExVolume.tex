\documentclass[handout]{ximera}
%\documentclass[10pt,handout,twocolumn,twoside,wordchoicegiven]{xercises}
%\documentclass[10pt,handout,twocolumn,twoside,wordchoicegiven]{xourse}

%\author{Steven Gubkin}
%\license{Creative Commons 3.0 By-NC}
\input{../preamble.tex}

\outcome{Practice with finding volume.}
   

\title[Exercises:]{Volume Exercises}

\begin{document}
\begin{abstract}
  Here we'll practice finding volume using integration.
\end{abstract}
\maketitle

%%Problem 1
\begin{exercise}
Consider the region bounded by $y = \frac{5x}{2}-x^2$ and
$y=\frac{x}{2}$.  What is the volume of the solid obtained by
revolving this region about the $x$-axis?

\begin{hint}
  First we find the points of intersections:
  \begin{align*}
    \frac{x}{2} &= \frac{5x}{2}-x^2\\
	0 &= 2x-x^2\\
	0&= x(2-x)
	\end{align*}

  So the points of intersection are $x=0$ and $x=2$.
\end{hint}

\begin{hint}
  By graphing the two functions, we can see that $\frac{5x}{2}-x^2$ is
  always greater than $\frac{x}{2}$ on the interval $[0,2]$.
\end{hint}

\begin{hint}
  We can decompose the solid into infinitesmal washers with width
  $\d x$, inner radius $\frac{x}{2}$ and outer radius
  $\frac{5x}{2}-x^2$.  The volume of each washer is
  $\pi((\frac{5x}{2}-x^2)^2-(\frac{x}{2})^2)\d x$.  Summing these
  volumes from $x=0$ to $x=2$, we obtain
  \[
  \textrm{Volume} = \int_0^2 \pi((\frac{5x}{2}-x^2)^2-(\frac{x}{2})^2) \d x
  \]
\end{hint}

\begin{hint}
  \begin{align*}
    \int_0^2 \pi((\frac{5x}{2}-x^2)^2-(\frac{x}{2})^2) \d x&= \pi \int_0^2 \pi(\frac{25x^2}{4}-5x^3+x^4)-(\frac{x^2}{4})) \d x\\
    &=\pi \int_0^2 12x^2-5x^3+x^4 \d x\\
    &= \eval{4x^3-\frac{5}{4}x^4+\frac{1}{5}x^5}_0^2\\
    &=4(2^3)-\frac{5}{4}2^4+\frac{1}{5}2^5\\
    &=\frac{92}{5}
  \end{align*}
\end{hint}

\begin{prompt}
  \[
	\textrm{Volume} = \answer{\frac{92}{5}}
	\]
\end{prompt}

\end{exercise}

%%Problem 2
\begin{exercise}
Consider the region bounded by $y =\sqrt{x-1}$ , the $x$-axis, and the
vertical line $x=2$.  What is the volume of the solid obtained by
revolving this region about the $y$-axis?
\begin{hint}
  Draw a picture!
\end{hint}

\begin{hint}
  Solving for $x$, we have $x = y^2+1$.  Note that $y$ ranges from $0$ to $1$ as $x$ goes from $1$ to $2$.
\end{hint}

\begin{hint}
  We can decompose the solid into infinitesmal washers with width
  $\d y$, inner radius $y^2+1$ and outer radius $2$.  The volume of each
  washer is $\pi((2)^2 - (y^2+1)^2)\d y$.  Summing these volumes from
  $y=0$ to $y=1$, we obtain
  \[
  \textrm{Volume} = \int_0^1 \pi((2)^2 - (y^2+1)^2)\d y
  \]
\end{hint}

\begin{hint}
  \begin{align*}
    \int_0^1\pi((2)^2 - (y^2+1)^2)\d y &=  \int_0^1 \pi(3-2y^2-y^4)\d y\\
    &= \pi \eval{\frac{-y^5}{5}+\frac{-2y^3}{3}+3y}_0^1\\
    &=\pi(\frac{-1}{5}+\frac{-2}{3}+3)\\
    &=\frac{32\pi}{15}
  \end{align*}
\end{hint}

\begin{prompt}
  \[
  \textrm{Volume} = \answer{\frac{32\pi}{15}}
  \]
\end{prompt}
\end{exercise}

%%Problem 3
\begin{exercise}
Consider the region bounded by $y =x^2$ and $y=4$.  A solid has this
region as its base, and the cross sections of the solid when cut with
planes parallel to the $(y,z)$-plane are all squares.  What is the area
of the solid?


\begin{hint}
  Draw a picture!  Note that the intersections occur at $x= \pm 2$
\end{hint}

\begin{hint}
  We can decompose the solid into square slabs with width $\d x$ and
  side lengths $4-x^2$.  The volume of each slab is $(4-x^2)\d x$.
  Summing these volumes from $x=-2$ to $x=2$, we obtain
	\[
	\textrm{Volume} = \int_{-2}^{2} (4-x^2)^2\d x
	\]
\end{hint}

\begin{hint}
  First note that this function is even, so we may use symmetry to rewrite the integral as $2\int_{0}^{2} (4-x^2)^2\d x$
  \begin{align*}
    2\int_{0}^{2} (4-x^2)^2\d x &=2 \int_{0}^{2} 16-8x^2+x^4 \d x \\
    &=2 \eval{16x-\frac{8x^3}{3}+\frac{x^5}{5}}_0^2 \\
    &=2(32-\frac{64}{3}+\frac{32}{5})\\
    &=\frac{512}{15}
  \end{align*}
\end{hint}

\begin{prompt}
  \[
  \textrm{Volume} = \answer{\frac{512}{15}}
  \]
\end{prompt}

\end{exercise}

%%Problem 4
\begin{exercise}
Consider the region bounded by $y = x$ and $y=\frac{x^3}{4}$ in the
first quadrant.  What is the volume of the solid obtained by
revolving this region about the line $y=-1$?

\begin{hint}
	Draw a picture!
\end{hint}

\begin{hint}
  First we find the points of intersectios.
  \begin{align*}
    x&= \frac{x^3}{4}\\
    4x &= x^3\\
    x^3-4x &=0\\
    x(x-2)(x+2) &=0
  \end{align*}

  So the points of intersection are $x=-2$, $x=0$, $x=2$.  Since we only care about the first quadrant, our bounds are from $x=0$ to $x=2$.
\end{hint}

\begin{hint}
  By graphing the two functions, we can see that $y=x$ is always greater than $y=\frac{x^3}{4}$ on the interval $[0,2]$.
\end{hint}

\begin{hint}
  We can decompose the solid into infinitesmal washers with width $dx$, inner radius $\frac{x^3}{4}+1$ and outer radius $x+1$. The volume of each washer is $\pi((x+1)^2-(\frac{x^3}{4}+1)^2)\d x$.  Summing these volumes from $x=0$ to $x=2$, we obtain
  
  \[
  \textrm{Volume} = \int_0^2 \pi((\frac{5x}{2}-x^2)^2-(\frac{x}{2})^2) \d x
  \]
\end{hint}


\begin{hint}
	By expanding this polynomial, we find that this evaluates to $\frac{74\pi}{21}$
\end{hint}

\begin{prompt}
  \[
  \textrm{Volume} = \answer{\frac{74\pi}{21}}
  \]
\end{prompt}

\end{exercise}

%%Problem 5
\begin{exercise}
Consider the region bounded by the lines $y = x$ , $y=\frac{x}{2}$,
$y=1$, and $y = 4$.  What is the volume of the solid obtained by
revolving this region about the line $x=-2$?

\begin{hint}
	Draw a picture!
\end{hint}

\begin{hint}
	Solving for $x$, we have $x=y$ and $x = 2y$.
\end{hint}

\begin{hint}
  We can decompose the solid into infinitesmal washers with width $\d
  y$, inner radius $y+2$ and outer radius $2y+2$. The volume of each
  washer is $\pi((2y+2)^2-(y+2)^2)\d y$.  Summing these volumes from
  $y=1$ to $y=4$, we obtain
  \[
  \textrm{Volume} = \int_1^4 \pi((2y+2)^2-(y+2)^2)\d y
  \]
\end{hint}


\begin{hint}
  By expanding this polynomial, we find that this evaluates to
  $93\pi$.
\end{hint}
\begin{prompt}
  \[
  \textrm{Volume} = \answer{93\pi}
  \]
\end{prompt}

\end{exercise}


%%Problem 6
\begin{exercise}
Consider the region bounded by the lines $y = \sin(x)$, $x=0$,
$x=\pi$, and the $x$-axis.  What is the volume of the solid obtained
by revolving this region about $x$-axis?

It will be useful to recall that $\sin^2(x) = \frac{1-\cos(2x)}{2}$.

\begin{hint}
  Draw a picture!
\end{hint}

\begin{hint}
  We can decompose the solid into infinitesmal disks with width
  $\d x$ and radius $\sin(x)$. The volume of each washer is $\pi
  \sin^2(x)\d x$.  Summing these volumes from $x=0$ to $x=\pi$,
  we obtain
  \[
  \textrm{Volume} = \int_0^\pi \pi \sin^2(x)\d x
  \]
\end{hint}

\begin{hint}
  Using the half angle reduction formula, and a substitution, we obtain that this evaluates to $\frac{\pi^2}{2}$
\end{hint}

\begin{prompt}
  \[
  \textrm{Volume} = \answer{\frac{\pi^2}{2}}
  \]
\end{prompt}

\end{exercise}













\end{document}
