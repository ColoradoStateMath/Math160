\documentclass[handout]{ximera}
%\documentclass[10pt,handout,twocolumn,twoside,wordchoicegiven]{xercises}
%\documentclass[10pt,handout,twocolumn,twoside,wordchoicegiven]{xourse}

%\author{Steven Gubkin}
%\license{Creative Commons 3.0 By-NC}
\input{../preamble.tex}

\outcome{Practice Limits.}

\title[Exercises:]{Limit Exercises}

\begin{document}
\begin{abstract}
  Here we'll practice limits.
\end{abstract}
\maketitle

%% Question 1 and 2 from exercises
\begin{exercise}
\[\lim_{x \to 0} \frac{\sin(2x)}{3x} = \answer{2/3}\]
\begin{hint}
	\[\frac{\sin(2x)}{3x} = \frac{\sin(2x)}{2x} \frac{2}{3}\]
\end{hint}
\end{exercise}

%% Question 3 from exercises
\begin{exercise}
\[\lim_{x \to 0} \frac{\sin(5x)}{\sin(2x)} = \answer{5/2}\]
\begin{hint}
	\[ \frac{\sin(5x)}{\sin(2x)} = \frac{\sin(5x)}{5x} \frac{2x}{\sin(2x)} \frac{5}{2}\]
\end{hint}
\end{exercise}

%% Question 4 from exercises
\begin{exercise}
\[\lim_{x \to 0} \frac{\cos(x)-1}{x} = \answer{0}\]
\begin{hint}
	\[\frac{\cos(x)-1}{x} = \frac{(\cos(x)-1)(\cos(x)+1)}{x(\cos(x)+1)}\]
\end{hint}
\begin{hint}
	Expanding the numerator, and using the pythagorean identity should get you to the solution.
\end{hint}
\end{exercise}

%% Question 5 from exercises
\begin{exercise}
\[\lim_{x \to 0} \frac{\cos(2x)-1}{x\sin(3x)} = \answer{2/3}\]
\begin{hint}
	Multiple numerator and denominator by $\cos(2x)+1$, expand the numerator, use pythagorean identity, and should be almost home free.
\end{hint}
\end{exercise}

%% Question 6 from exercises
\begin{exercise}
	\[\lim_{h \to 0} \frac{\sqrt{2+h} - \sqrt{2}}{h} = \answer{1/(2\sqrt{2})}\]
	\begin{hint}
		Multiply numerator and denominator by $\sqrt{2+h}+\sqrt{2}$.
	\end{hint}
\end{exercise}

%% Question 8 from exercises
\begin{exercise}
If I want $\lim_{x \to 0} \frac{1-\cos(ax)}{x\sin(2x)} = 4$ what should I pick for $a$?
	\[a = \answer{8}\]
	\begin{hint}
		Multiply numerator and denominator by $1+\cos(ax)$, expand the numerator, and use the pythagorean identity.  This will let you find the limit in terms of $a$.  Then solve the equation to find the value of $a$.
	\end{hint}
\end{exercise}

%% Question 9 from exercises
\begin{exercise}
	\begin{warning}
		This problem is pretty hard.
	\end{warning}
	\[\lim_{h \to 0} \frac{\sin(1+h) - 2\sin(1)+\sin(1-h)}{h^2} = \answer{-\sin(1)}\]
	\begin{hint}
		Use the angle sum formula for sine on both $\sin(1+h)$ and $\sin(1-h)$
	\end{hint}
\end{exercise}









\end{document}
