\documentclass{ximera}
\author{Steven Gubkin}
\license{Creative Commons 3.0 By-NC}
\input{../preamble.tex}

\outcome{Practice Limits.}


\title[Exercises:]{Limits}

\begin{document}
\begin{abstract}
  Here we'll practice limits.
\end{abstract}
\maketitle


\begin{exercise}
\outcome{Calculate limits of the form zero over zero.}
\[\lim_{x \to 0} \frac{\sin(2x)}{3x} = \answer{2/3}\]
\begin{hint}
	\[\frac{\sin(2x)}{3x} = \frac{\sin(2x)}{2x} \frac{2}{3}\]
\end{hint}
\end{exercise}


\begin{exercise}
\outcome{Calculate limits of the form zero over zero.}
\[\lim_{x \to 0} \frac{\sin(2x)}{3x} = \answer{2/3}\]
\begin{hint}
	\[\frac{\sin(2x)}{3x} = \frac{\sin(2x)}{2x} \frac{2}{3}\]
\end{hint}
\end{exercise}



\begin{exercise}
\outcome{Calculate limits of the form zero over zero.}
	\begin{warning}
		This problem is pretty hard.
	\end{warning}
	\[\lim_{h \to 0} \frac{\sin(1+h) - 2\sin(1)+\sin(1-h)}{h^2} = \answer{-\sin(1)}\]
	\begin{hint}
		Use the angle sum formula for sine on both $\sin(1+h)$ and $\sin(1-h)$
	\end{hint}
\end{exercise}









\end{document}
