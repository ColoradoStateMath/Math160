\documentclass[handout]{ximera}
%\documentclass[10pt,handout,twocolumn,twoside,wordchoicegiven]{xercises}
%\documentclass[10pt,handout,twocolumn,twoside,wordchoicegiven]{xourse}

%\author{Steven Gubkin}
%\license{Creative Commons 3.0 By-NC}
\input{../preamble.tex}

\outcome{Practice Limits.}
   

\title[Exercises:]{Definition of the Derivative}

\begin{document}
\begin{abstract}
  Here we'll practice finding the derivative using limits.
\end{abstract}
\maketitle

%%SIMPLE DERIVS LIMIT 10
\begin{exercise}
Let $F(w) = w^2$. Compute

\[
\lim_{ h\to 0 } 
\frac{F(2+h)-F(2)}{h} \begin{prompt}=\answer{10}\end{prompt}
\]
\end{exercise}

%%SIMPLE DERIVS LIMIT 10
\begin{exercise}
Let $F(y) = 2y-y^2$. Compute

\[
\lim_{ h\to 0 } 
\frac{F(3+h)-F(3)}{h} \begin{prompt}=\answer{-4}\end{prompt}
\]
\end{exercise}

%%SIMPLE DERIVS LIMIT 1
\begin{exercise}
Let $y(x) = (x+4)^2$. Compute

\[
\lim_{ h\to 0 } \frac{y(1+h)-y(1)}{h} \begin{prompt}=\answer{-4}\end{prompt}
\]
\end{exercise}


%%SIMPLE DERIVS LIMIT 1
\begin{exercise}
Let $y(\theta) = (\theta -3)^2$. Compute

\[
\lim_{ h\to 0 } \frac{y(1+h)-y(1)}{h} \begin{prompt}=\answer{-4}\end{prompt}
\]
\end{exercise}


%%SIMPLE DERIVS LIMIT 3
\begin{exercise}
Let $f(x) = \sqrt{x+3}$. Compute

\[
\lim_{ h\to 0 } 
\frac{f(-1+h)-f(-1)}{h} \begin{prompt}=\answer{\frac{1}{2 \sqrt{2}}}\end{prompt}
\]
\end{exercise}


%%SIMPLE DERIVS LIMIT 6
\begin{exercise}
Let $G(x) = \frac{1}{x-4}$. Compute

\[
\lim_{ h\to 0 } \frac{G(-1+h)-G(-1)}{h} \begin{prompt}=\answer{-\frac{1}{25}}\end{prompt}
\]
\end{exercise}




























\end{document}
