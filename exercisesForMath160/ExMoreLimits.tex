\documentclass[handout]{ximera}
%\documentclass[10pt,handout,twocolumn,twoside,wordchoicegiven]{xercises}
%\documentclass[10pt,handout,twocolumn,twoside,wordchoicegiven]{xourse}

%\author{Steven Gubkin}
%\license{Creative Commons 3.0 By-NC}
\input{../preamble.tex}

\outcome{Exercises: More Practice with Limits}
   

\title{More Limits}

\begin{document}
\begin{abstract}
  Here we'll practice finding limits.
\end{abstract}
\maketitle

%%PROBLEM 1
\begin{exercise}
Find
\[
\lim_{\theta\to\infty}\left(\cos(\theta)\right)
\begin{prompt}
= \answer{DNE}
\end{prompt}
\]

\end{exercise}


%%PROBLEM 2
\begin{exercise}
Find
\[
\lim_{x\to6}\left(\frac{\left|6-x\right|}{6-x}\right)
\begin{prompt}
= \answer{DNE}
\end{prompt}
\]

\end{exercise}

%%PROBLEM 3
\begin{exercise}
Find
\[
\lim_{x\to0}\left(\frac{\cos^{2}(x)-1}{\cos(x)-1}\right)
\begin{prompt}
= \answer{2}
\end{prompt}
\]

\begin{hint}
Recall that for any two numbers, $a$ and $b$,  $a^2-b^2=(a-b)(a+b)$. Factor the numerator of the expression in this manner.
\end{hint}
\begin{hint}
$\cos^{2}(x)-1=(\cos(x)-1)(\cos(x)+1)$ so $\frac{(\cos(x)-1)(\cos(x)+1)}{\cos(x)-1}=\cos(x)+1$ for $x\ne2\pi{n}$ for any integer $n$ (notice that if $x=2\pi n$, then $\cos(2\pi n)-1=0$ in the denominator) and $\lim_{x\to0}\left(\frac{\cos^{2}(x)-1}{\cos(x)-1}\right)=\lim_{x\to0}\left(\cos(x)+1\right)$ and this is $\lim_{x\to0}\left(\cos(x)\right)+\lim_{x\to0}\left(1\right)=2$.

\end{hint}
\end{exercise}

%%PROBLEM 4
\begin{exercise}
Find
\[
\lim_{x\to-\infty}\left(\frac{2x^3-3x^2+4}{x^3+3x^2-1}\right)
\begin{prompt}
= \answer{2}
\end{prompt}
\]

\end{exercise}


%%PROBLEM 5
\begin{exercise}
Suppose that $\lim_{z\to5}G(z)=3$, $\lim_{z\to5}C(z)=-4$, and $\lim_{z\to5}c(z)=1$. Compute the limit

\[
\lim_{z\to 5 } \frac{G(z)}{C(z)-c(z)}\begin{prompt} = \answer{-\frac{3}{5}}\end{prompt}
\]
\end{exercise}

%%PROBLEM 6
\begin{exercise}
\[
\lim_{n\to -5 } \frac{\sqrt{4-n}-3}{n+5}\begin{prompt} = \answer{-\frac{1}{6}}\end{prompt}
\]
\begin{hint}
Multiply by $\frac{\sqrt{4-n}+3}{\sqrt{4-n}+3}$.
\end{hint}
\end{exercise}

%%PROBLEM 7
\begin{exercise}
\[
\lim_{n\to 5 } \frac{\sqrt{n+5}-\sqrt{10}}{n-5}\begin{prompt} = \answer{\frac{1}{2 \sqrt{10}}}\end{prompt}
\]
\begin{hint}
Multiply by $\frac{\sqrt{n+5}+\sqrt{10}}{\sqrt{n+5}+\sqrt{10}}$.
\end{hint}
\end{exercise}


%%PROBLEM 8
\begin{exercise}
\[
\lim_{t\to 3 } \frac{t-3}{\sqrt{t-2}-1}\begin{prompt} = \answer{2}\end{prompt}
\]
\begin{hint}
Multiply by $\frac{\sqrt{t-2}+1}{\sqrt{t-2}+1}$.
\end{hint}
\end{exercise}

%%PROBLEM 9
\begin{exercise}
\[\lim_{x \to 0} \frac{\sin(2x)}{3x} = \answer{2/3}\]

\begin{hint}
	\[\frac{\sin(2x)}{3x} = \frac{\sin(2x)}{2x} \frac{2}{3}\]
\end{hint}


\end{exercise}

%%PROBLEM 10
\begin{exercise}
\[
\lim_{v\to -5 } \frac{v^2+6 v+5}{v+5}\begin{prompt} = \answer{-4}\end{prompt}
\]
\begin{hint}
Try to factor either the numerator or the denominator.
\end{hint}
\end{exercise}

%%PROBLEM 11
\begin{exercise}
\[
\lim_{z\to -4 } \frac{z^2+6 z+8}{z+4}\begin{prompt} = \answer{-2}\end{prompt}
\]
\begin{hint}
Try to factor either the numerator or the denominator.
\end{hint}
\end{exercise}

%%PROBLEM 12
\begin{exercise}
\[
\lim_{z\to -2 } \frac{z+2}{-2 z^2-2 z+4}\begin{prompt} = \answer{\frac{1}{6}}\end{prompt}
\]
\begin{hint}
Try to factor either the numerator or the denominator.
\end{hint}
\end{exercise}


%%PROBLEM 13
\begin{exercise}
\[
\lim_{\theta\to -1 } \frac{\theta +1}{-2 \theta ^2-6 \theta -4}\begin{prompt} = \answer{-\frac{1}{2}}\end{prompt}
\]
\begin{hint}
Try to factor either the numerator or the denominator.
\end{hint}
\end{exercise}










\end{document}
