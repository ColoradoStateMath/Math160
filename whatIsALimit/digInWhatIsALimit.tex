\documentclass{ximera}

\input{../preamble.tex}

\outcome{Consider function values nearer and nearer to a given input value.}
\outcome{Understand the concept of a limit.}
\outcome{Limits as understanding local behavior of functions.}
\outcome{Calculate limits from a graph (or state that the limit does not exist).}
\outcome{Define a one-sided limit.}


\title[Dig-In:]{What is a limit?}
\begin{document}
\begin{abstract}
  We introduce limits.
\end{abstract}
\maketitle

\section{The basic idea}

Consider the function
\[
f(x) = \frac{\sin(x)}{x}.
\]
While $f(x)$ is undefined at $x=0$, we can still plot $f(x)$ at other
values near $x = 0$.
\begin{image}
  \begin{tikzpicture}
    \begin{axis}[
        xmin=-6.75,xmax=6.75,ymin=-1.5,ymax=1.5,
        axis lines=center,
        xtick={-6.28, -4.71, -3.14, -1.57, 0, 1.57, 3.142, 4.71, 6.28},
        xticklabels={$-2\pi$,$-3\pi/2$,$-\pi$, $-\pi/2$, $0$, $\pi/2$, $\pi$, $3\pi/2$, $2\pi$},
        ytick={-1,1},
        %ticks=none,
        width=6in,
        height=3in,
        unit vector ratio*=1 1 1,
        xlabel=$x$, ylabel=$y$,
        every axis y label/.style={at=(current axis.above origin),anchor=south},
        every axis x label/.style={at=(current axis.right of origin),anchor=west},
      ]        
      \addplot [very thick, penColor, samples=100,smooth, domain=(-6.75:6.75)] {sin(deg(x))/x};
      
      \addplot[color=penColor,fill=white,only marks,mark=*] coordinates{(0,1)};  %% open hole          
    \end{axis}
  \end{tikzpicture}
\end{image}

\begin{question}
  Use the graph of $f(x) = \frac{\sin(x)}{x}$ above to answer the
  following question: What is $f(0)$?
  \begin{multipleChoice}
    \choice{$0$}
    \choice{$f(0)$}
    \choice{$1$}
    \choice[correct]{$f(0)$ is undefined}
    \choice{it is impossible to say}
  \end{multipleChoice}
\end{question}

Nevertheless, we can see that as $x$ approaches zero, $f(x)$
approaches one. From this setting we come to our definition of a
limit.

\begin{definition}
  Intuitively,
  \begin{center}
    the \dfn{limit} of $f(x)$ as $x$ approaches $a$ is $L$,
  \end{center}
  written
  \[
  \lim_{x\to a} f(x) = L,
  \]
  if the value of $f(x)$ can be made as close as one wishes to $L$ for
  all $x$ sufficiently close, but not equal to, $a$.
\end{definition}

\begin{question}
  Use the graph of $f(x) = \frac{\sin(x)}{x}$ above to finish the following statement: ``A good guess is that\dots''
  \begin{multipleChoice}
    \choice[correct]{$\lim_{x\to 0}\frac{\sin(x)}{x} = 1$.}
    \choice{$\displaystyle\lim_{x\to 1}\frac{\sin(x)}{x} = 0$.}
    \choice{$\displaystyle\lim_{x\to 1}f(x) = \frac{\sin(1)}{1}$.}
    \choice{$\displaystyle\lim_{x\to 0}f(x) = \frac{\sin(0)}{0}=\infty$.}
  \end{multipleChoice}
\end{question}


\begin{question}
  Consider the following graph of $y=f(x)$
\begin{image}
          
\begin{tikzpicture}
	\begin{axis}[
            xmin=-4, xmax=4, ymin=-3,ymax=3,    
            unit vector ratio*=1 1 1,
            axis lines =middle, xlabel=$x$, ylabel=$y$,
            every axis y label/.style={at=(current axis.above origin),anchor=south},
            every axis x label/.style={at=(current axis.right of origin),anchor=west},
            xtick={-4,...,4}, ytick={-3,...,3},
            %xticklabels={-4,,-2,,0,,2,,4,,6}, yticklabels={,-2,,0,,2,,4,,6,,8,,10},
            grid=major,width=4in,
            grid style={dashed, gridColor},
          ]
	  \addplot[color=penColor,very thick] plot coordinates
                  {(-4,-1) (-1,2) (1,-2) (4,1)};

          \addplot[color=penColor,fill=penColor,only marks,mark=*] coordinates{(0,-2)};  %% closed hole
          \addplot[color=penColor,fill=background,only marks,mark=*] coordinates{(0,0)};  %% open hole
          \addplot[color=penColor,fill=background,only marks,mark=*] coordinates{(1,-2)};  %% open hole
        \end{axis}
\end{tikzpicture}
\end{image}
  Use the graph to evaluate the following. Write DNE if the value does
  not exist.
  \begin{enumerate}
    \item $f(-2) \begin{prompt}=\answer{1}\end{prompt}$
    \item $\displaystyle\lim_{x\to -2}f(x)\begin{prompt}=\answer{1}\end{prompt}$
    \item $f(-1) \begin{prompt}=\answer{2}\end{prompt}$
    \item $\displaystyle\lim_{x\to -1}f(x) \begin{prompt}=\answer{2}\end{prompt}$
    \item $f(0) \begin{prompt}=\answer{-2}\end{prompt}$
    \item $\displaystyle\lim_{x\to 0} f(x) \begin{prompt}=\answer{0}\end{prompt}$
    \item $f(1) \begin{prompt}=\answer{DNE}\end{prompt}$
    \item $\displaystyle\lim_{x\to 1} f(x) \begin{prompt}=\answer{-2}\end{prompt}$
  \end{enumerate}
\end{question}





\section{Limits might not exist}

Limits might not exist. Let's see how this happens.


\begin{example}
  Consider the graph of $f(x) = \lfloor x\rfloor$.
\begin{image}
\begin{tikzpicture}
	\begin{axis}[
            domain=-2:4,
            axis lines =middle, xlabel=$x$, ylabel=$y$,
            every axis y label/.style={at=(current axis.above origin),anchor=south},
            every axis x label/.style={at=(current axis.right of origin),anchor=west},
            clip=false,
          ]
          \addplot [very thick, penColor, domain=(-2:-1)] {-2};
          \addplot [very thick, penColor, domain=(-1:0)] {-1};
          \addplot [very thick, penColor, domain=(0:1)] {0};
          \addplot [very thick, penColor, domain=(1:2)] {1};
          \addplot [very thick, penColor, domain=(2:3)] {2};
          \addplot [very thick, penColor, domain=(3:4)] {3};
          \addplot[color=penColor,fill=penColor,only marks,mark=*] coordinates{(-2,-2)};  %% closed hole          
          \addplot[color=penColor,fill=penColor,only marks,mark=*] coordinates{(-1,-1)};  %% closed hole          
          \addplot[color=penColor,fill=penColor,only marks,mark=*] coordinates{(0,0)};  %% closed hole          
          \addplot[color=penColor,fill=penColor,only marks,mark=*] coordinates{(1,1)};  %% closed hole          
          \addplot[color=penColor,fill=penColor,only marks,mark=*] coordinates{(2,2)};  %% closed hole  
          \addplot[color=penColor,fill=penColor,only marks,mark=*] coordinates{(3,3)};  %% closed hole                  
          \addplot[color=penColor,fill=background,only marks,mark=*] coordinates{(-1,-2)};  %% open hole
          \addplot[color=penColor,fill=background,only marks,mark=*] coordinates{(0,-1)};  %% open hole
          \addplot[color=penColor,fill=background,only marks,mark=*] coordinates{(1,0)};  %% open hole
          \addplot[color=penColor,fill=background,only marks,mark=*] coordinates{(2,1)};  %% open hole
          \addplot[color=penColor,fill=background,only marks,mark=*] coordinates{(3,2)};  %% open hole
          \addplot[color=penColor,fill=background,only marks,mark=*] coordinates{(4,3)};  %% open hole
        \end{axis}
\end{tikzpicture}
%% \caption{A plot of $f(x)=\lfloor x\rfloor$. Note, no matter which
%%   $\delta>0$ is chosen, we can only at best bound $f(x)$ in the
%%   interval $[1,2]$. With the example of $f(x) = \lfloor x \rfloor$, we
%%   see that taking limits is truly different from evaluating
%%   functions.}
\end{image}
Explain why the limit
\[
\lim_{x\to 2} f(x)
\]
does not exist.
\begin{explanation}
The function $\lfloor x \rfloor$ is the function that returns the
greatest integer less than or equal to $x$. Recall that
\[
\lim_{x\to 2} \lfloor x \rfloor  = L
\]
if $\lfloor x\rfloor$ can be made arbitrarily close to $L$ by making
$x$ sufficiently close, but not equal to, $2$. So let's examine $x$
near, but not equal to, $2$. Now the question is: What is $L$?



If this limit exists, then we should be able to look sufficiently
close, but not at, $x=2$, and see that $f$ is approaching some number.
Let's look at a graph:
\begin{image}
\begin{tikzpicture}
	\begin{axis}[
            domain=1.5:2.5,
            ymin=-.5,ymax=3.5,
            axis lines =middle, xlabel=$x$, ylabel=$y$,
            every axis y label/.style={at=(current axis.above origin),anchor=south},
            every axis x label/.style={at=(current axis.right of origin),anchor=west},
          ]
	  \addplot [very thick, penColor, domain=(1:2)] {1};
          \addplot [very thick, penColor, domain=(2:3)] {2};
          \addplot[color=penColor,fill=penColor,only marks,mark=*] coordinates{(2,2)};  %% closed hole  
          \addplot[color=penColor,fill=background,only marks,mark=*] coordinates{(2,1)};  %% open hole
          
          \addplot[(-),color=penColor2,ultra thick] plot coordinates {(1.75,0) (2.25,0)};

          \addplot[color=penColor2,fill=background,only marks,mark=*] coordinates{(2,0)};  %% open hole
          \addplot[textColor,dashed] plot coordinates {(1.9,0) (1.9,1)};
        \end{axis}
\end{tikzpicture}
\end{image}
If we look closer and closer to $x=2$ (on the left of $x=2$) we see
that $f(x) = 1$. However, if we look closer and closer to $x=2$ (on
the right of $x=2$) we see
\begin{image}
\begin{tikzpicture}
	\begin{axis}[
            domain=1.5:2.5,
            ymin=-.5,ymax=3.5,
            axis lines =middle, xlabel=$x$, ylabel=$y$,
            every axis y label/.style={at=(current axis.above origin),anchor=south},
            every axis x label/.style={at=(current axis.right of origin),anchor=west},
          ]
	  \addplot [very thick, penColor, domain=(1:2)] {1};
          \addplot [very thick, penColor, domain=(2:3)] {2};
          \addplot[color=penColor,fill=penColor,only marks,mark=*] coordinates{(2,2)};  %% closed hole  
          \addplot[color=penColor,fill=background,only marks,mark=*] coordinates{(2,1)};  %% open hole
          
          \addplot[(-),color=penColor2,ultra thick] plot coordinates {(1.75,0) (2.25,0)};

          \addplot[color=penColor2,fill=background,only marks,mark=*] coordinates{(2,0)};  %% open hole
          \addplot[textColor,dashed] plot coordinates {(2.1,0) (2.1,2)};
        \end{axis}
\end{tikzpicture}
\end{image}
So just to the right of $x=2$, $f(x)=2$. We cannot find a single
number that $f(x)$ approaches as $x$ approaches $x=2$, and so the
limit does not exists.
\end{explanation}
\end{example}

Tables can be used to help guess limits, but one must be careful.

\begin{question}
  Consider $f(x) = \sin\left(\frac{\pi}{x}\right)$. Fill in the 
  tables below:
  \[
  \begin{array}{l|l}
    x      & f(x)      \\ \hline
    0.1    & \begin{prompt}\answer{0}\end{prompt}\\
    0.01   & \begin{prompt}\answer{0}\end{prompt}\\
    0.001  & \begin{prompt}\answer{0}\end{prompt}\\
    0.0001 & \begin{prompt}\answer{0}\end{prompt} 
  \end{array}
  \qquad\text{and}\qquad
  \begin{array}{l|l}
    x      & f(x)            \\ \hline
    0.3    &  \begin{prompt}\answer{-.866}\end{prompt} \\
    0.03   &  \begin{prompt}\answer{-.866}\end{prompt} \\
    0.003  &  \begin{prompt}\answer{-.866}\end{prompt} \\
    0.0003 &  \begin{prompt}\answer{-.866}\end{prompt}
  \end{array}
  \]
  What do the tables tell us about
  \[
  \lim_{x\to 0}\sin\left(\frac{\pi}{x}\right)?
  \]
  \begin{multipleChoice}
    \choice{$\displaystyle\lim_{x\to 0}\sin\left(\frac{\pi}{x}\right) = 0$}
    \choice{$\displaystyle\lim_{x\to 0}\sin\left(\frac{\pi}{x}\right)=1$}
    \choice{$\displaystyle\lim_{x\to 0}\sin\left(\frac{\pi}{x}\right) = -.866$}
    \choice{$\displaystyle\lim_{x\to 0}\sin\left(\frac{\pi}{x}\right) = -.433$}
    \choice[correct]{it is unclear what the tables are telling us about the limit}
  \end{multipleChoice}
  \begin{feedback}
    Neither tables nor graphs can ever tell us for certain what a
    limit is. However, sometimes they can help ``guess'' the limit. In
    this case the graph of $f(x)$ is somewhat more helpful:
    \begin{image}
      \begin{tikzpicture}
	\begin{axis}[
            domain=-.2:.2,    
            samples=500,
            axis lines =middle, xlabel=$x$, ylabel=$y$,
            yticklabels = {}, 
            every axis y label/.style={at=(current axis.above origin),anchor=south},
            every axis x label/.style={at=(current axis.right of origin),anchor=west},
            clip=false,
          ]
	  \addplot [very thick, penColor, smooth, domain=(-.2:-.02)] {sin(deg(pi/x))};
          \addplot [very thick, penColor, smooth, domain=(.02:.2)] {sin(deg(pi/x))};
	  \addplot [color=penColor, fill=penColor, very thick, smooth,domain=(-.05:.05)] {1} \closedcycle;
          \addplot [color=penColor, fill=penColor, very thick, smooth,domain=(-.05:.05)] {-1} \closedcycle;
        \end{axis}
      \end{tikzpicture}
    \end{image}
   We see that $f(x)$ oscillates ``wildly'' as $x$ approaches $0$, and hence does not approach any one number.
  \end{feedback}
\end{question}



\section{One-sided limits}


While we have seen that $\displaystyle\lim_{x\to 2}\lfloor x\rfloor$ does not
exist, more can still be said.



\begin{definition}
  Intuitively,
  \begin{center}
    the \dfn{limit from the right} of $f$ as $x$ approaches $a$ is
    $L$,
  \end{center}
  written
  \[
  \lim_{x\to a^+} f(x) = L,
  \]
  if the value of $f(x)$ can be made as close as one wishes to $L$ for
  all $x>a$ sufficiently close, but not equal to, $a$.
  
  Similarly,
  \begin{center}
    the \dfn{limit from the left} of $f(x)$ as $x$ approaches $a$ is
    $L$,
  \end{center}
  written
  \[
  \displaystyle\lim_{x\to a^-} f(x) = L,
  \]
  if the value of $f(x)$ can be made as close as one wishes to $L$ for
  all $x<a$ sufficiently close, but not equal to, $a$.
\end{definition}


\begin{example}
Compute:
\[
\displaystyle\lim_{x\to 2^-} f(x)\qquad\text{and}\qquad \lim_{x\to 2^+} f(x)
\]
by using the graph below
\begin{image}
\begin{tikzpicture}
	\begin{axis}[
            domain=-2:4,
            axis lines =middle, xlabel=$x$, ylabel=$y$,
            every axis y label/.style={at=(current axis.above origin),anchor=south},
            every axis x label/.style={at=(current axis.right of origin),anchor=west},
            clip=false,
          ]
          \addplot [very thick, penColor, domain=(-2:-1)] {-2};
          \addplot [very thick, penColor, domain=(-1:0)] {-1};
          \addplot [very thick, penColor, domain=(0:1)] {0};
          \addplot [very thick, penColor, domain=(1:2)] {1};
          \addplot [very thick, penColor, domain=(2:3)] {2};
          \addplot [very thick, penColor, domain=(3:4)] {3};
          \addplot[color=penColor,fill=penColor,only marks,mark=*] coordinates{(-2,-2)};  %% closed hole          
          \addplot[color=penColor,fill=penColor,only marks,mark=*] coordinates{(-1,-1)};  %% closed hole          
          \addplot[color=penColor,fill=penColor,only marks,mark=*] coordinates{(0,0)};  %% closed hole          
          \addplot[color=penColor,fill=penColor,only marks,mark=*] coordinates{(1,1)};  %% closed hole          
          \addplot[color=penColor,fill=penColor,only marks,mark=*] coordinates{(2,2)};  %% closed hole  
          \addplot[color=penColor,fill=penColor,only marks,mark=*] coordinates{(3,3)};  %% closed hole                  
          \addplot[color=penColor,fill=background,only marks,mark=*] coordinates{(-1,-2)};  %% open hole
          \addplot[color=penColor,fill=background,only marks,mark=*] coordinates{(0,-1)};  %% open hole
          \addplot[color=penColor,fill=background,only marks,mark=*] coordinates{(1,0)};  %% open hole
          \addplot[color=penColor,fill=background,only marks,mark=*] coordinates{(2,1)};  %% open hole
          \addplot[color=penColor,fill=background,only marks,mark=*] coordinates{(3,2)};  %% open hole
          \addplot[color=penColor,fill=background,only marks,mark=*] coordinates{(4,3)};  %% open hole
        \end{axis}
\end{tikzpicture}
%% \caption{A plot of $f(x)=\lfloor x\rfloor$. Note, no matter which
%%   $\delta>0$ is chosen, we can only at best bound $f(x)$ in the
%%   interval $[1,2]$. With the example of $f(x) = \lfloor x \rfloor$, we
%%   see that taking limits is truly different from evaluating
%%   functions.}
\end{image}
\begin{explanation}
  From the graph we can see that as $x$ approaches $2$ from the left, $\lfloor x\rfloor$ remains at $y=1$ up  until the exact point that $x=2$. Hence
  \[
  \displaystyle\lim_{x\to 2^-} f(x)=1.
  \]
  Also from the graph we can see that as $x$ approaches $2$ from the
  right, $\lfloor x\rfloor$ remains at $y=2$ up to $x=2$. Hence
  \[
  \displaystyle\lim_{x\to 2^+} f(x)=2.
  \]
\end{explanation}
\end{example}



\section{When you put this all together}

One-sided limits help us talk about limits.
\begin{theorem}\index{limit}\index{one-sided limit}
  A limit
  \[
  \lim_{x \to a} f(x)
  \]
  exists if and only if
  \begin{itemize}
  \item $\displaystyle\lim_{x \to a^-} f(x)$ exists
  \item $\displaystyle\lim_{x \to a^+} f(x)$ exists
  \item $\displaystyle\lim_{x \to a^-} f(x) = \lim_{x \to a^+} f(x)$
  \end{itemize}
  In this case, $\displaystyle\lim_{x \to a} f(x)$ is equal to the common
  value of the two one sided limits.
\end{theorem}


\begin{question}
  Evaluate the expressions by referencing the graph below. Write DNE
  if the limit does not exist.
\begin{image}
\begin{tikzpicture}
	\begin{axis}[
            domain=-4:6, xmin=-4, xmax=6, ymin=-3,ymax=10,    
            unit vector ratio*=1 1 1,
            axis lines =middle, xlabel=$x$, ylabel=$y$,
            every axis y label/.style={at=(current axis.above origin),anchor=south},
            every axis x label/.style={at=(current axis.right of origin),anchor=west},
            xtick={-4,...,6}, ytick={-3,...,10},
            xticklabels={-4,,-2,,0,,2,,4,,6}, yticklabels={,-2,,0,,2,,4,,6,,8,,10},
            grid=major,width=4in,
            grid style={dashed, gridColor},
          ]
	  \addplot [very thick, penColor, smooth, domain=(-4:-2)] {6};
	  \addplot [very thick, penColor, smooth, domain=(-2:0)] {x^2-2};
          \addplot [very thick, penColor, smooth, domain=(0:2)] {(x-1)^3+3*(x-1)+3};
          \addplot [very thick, penColor, smooth, domain=(2:6)] {(x-4)^3+8};
          \addplot[color=penColor,fill=background,only marks,mark=*] coordinates{(-2,6)};  %% open hole
          \addplot[color=penColor,fill=background,only marks,mark=*] coordinates{(-2,2)};  %% open hole
          \addplot[color=penColor,fill=background,only marks,mark=*] coordinates{(0,-2)};  %% open hole
          \addplot[color=penColor,fill=background,only marks,mark=*] coordinates{(0,-1)};  %% open hole
          \addplot[color=penColor,fill=background,only marks,mark=*] coordinates{(2,0)};  %% open hole
          \addplot[color=penColor,fill=penColor,only marks,mark=*] coordinates{(-2,8)};  %% closed hole
          \addplot[color=penColor,fill=penColor,only marks,mark=*] coordinates{(0,-1.5)};  %% closed hole
          \addplot[color=penColor,fill=penColor,only marks,mark=*] coordinates{(2,7)};  %% closed hole
        \end{axis}
\end{tikzpicture}
\end{image}
\begin{enumerate}
\item $\displaystyle\lim_{x\to 4} f(x) \begin{prompt}=\answer{8}\end{prompt}$  
\item $\displaystyle\lim_{x\to -3} f(x)\begin{prompt}=\answer{6}\end{prompt}$  
\item $\displaystyle\lim_{x\to 0} f(x) \begin{prompt}=\answer{DNE}\end{prompt}$ 
\item $\displaystyle\lim_{x\to 0^-} f(x) \begin{prompt}=\answer{-2}\end{prompt}$  
\item $\displaystyle\lim_{x\to 0^+} f(x) \begin{prompt}=\answer{-1}\end{prompt}$  
\item $f(-2) \begin{prompt}=\answer{8}\end{prompt}$  
\item $\displaystyle\lim_{x\to 2^-} f(x) \begin{prompt}=\answer{7}\end{prompt}$  
\item $\displaystyle\lim_{x\to -2^-} f(x) \begin{prompt}=\answer{6}\end{prompt}$  
\item $\displaystyle\lim_{x\to 0} f(x+1) \begin{prompt}=\answer{3}\end{prompt}$  
\item $f(0) \begin{prompt}=\answer{-3/2}\end{prompt}$ 
\item $\displaystyle\lim_{x\to 1^-} f(x-4) \begin{prompt}=\answer{6}\end{prompt}$  
\item $\displaystyle\lim_{x\to 0^+} f(x-2) \begin{prompt}=\answer{2}\end{prompt}$
\end{enumerate}
\end{question}



\end{document}
