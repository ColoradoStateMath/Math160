\documentclass{ximera}

\input{../preamble.tex}

\title[Dig-In:]{Derivatives of trigonometric functions}

\outcome{Apply chain rule to relate quantities expressed with different units.}
\outcome{Compute derivatives of trigonometric functions.}

\begin{document}
\begin{abstract}
  We use the chain rule to unleash the derivatives of the
  trigonometric functions.
\end{abstract}
\maketitle


Up until this point of the course we have been ignoring a large class
of functions: Trigonometric functions other than $\sin(x)$. We know
that
\[
\ddx \sin(x) = \cos(x).
\]
Armed with this fact we will discover the derivatives of all of the
standard trigonometric functions.



\begin{theorem}[The derivative of cosine]\index{derivative!of cosine}
\[
\ddx \cos(x) = -\sin(x).
\]
\begin{explanation}
Recall that
\begin{itemize}
\item $\cos(x) = \sin\left(\frac{\pi}{2}-x\right)$, and
\item $\sin(x) = \cos\left(\frac{\pi}{2}-x\right)$.
\end{itemize}
Now
\begin{align*}
\ddx \cos(x) &= \ddx \sin\left(\frac{\pi}{2}-x\right)\\
&=-\cos\left(\frac{\pi}{2}-x\right) \\
&= -\sin(x).
\end{align*}
\end{explanation}
\end{theorem}

\begin{example}
Compute:
\[
\eval{\ddx \cos \left( \frac{x^3}{2} \right)}_{x=\sqrt[3]{\pi}}
\]
\begin{explanation}
Now that we know the derivative of cosine, we may combine this with the
chain rule, so we have that
\[
\ddx \cos \left( \frac{x^3}{2} \right) = \answer[given]{\frac{3 x^2}{2}} \left(- \sin \left( \frac{x^3}{2} \right) \right)
\]
and so
\[
\eval{\ddx \cos \left( \frac{x^3}{2} \right)}_{x=\sqrt[3]{\pi}}
\]
\begin{align*}
  &= \eval{\left( \frac{3}{2} x^2 \left(- \sin \left( \frac{x^3}{2}
    \right) \right) \right)}_{x=\sqrt[3]{\pi}} \\
  &= - \frac{3}{2}(\sqrt[3]{\pi})^2 \sin \left( \frac{\pi}{2} \right) \\
  &= -\frac{3}{2} \pi^{\frac{2}{3}} \cdot \answer[given]{1} \\
  &=\answer[given]{\frac{-3 \pi^{\frac{2}{3}}}{2}}.
\end{align*}
\end{explanation}
\end{example}


Next we have:

\begin{theorem}[The derivative of tangent]\index{derivative!of tangent}
\[
\ddx \tan(x) = \sec^2(x).
\]

\begin{explanation}
We'll rewrite $\tan(x)$ as $\frac{\sin(x)}{\cos(x)}$ and use the
quotient rule. Write with me:
\begin{align*}
\ddx\tan(x) &= \ddx\frac{\sin(x)}{\cos(x)}\\
&=\frac{\cos^2(x) + \answer[given]{\sin^2(x)}}{\cos^2(x)}\\
&=\frac{\answer[given]{1}}{\cos^2(x)}\\
&=\sec^2(x).
\end{align*}
\end{explanation}
\end{theorem}

\begin{example}
Compute:
\[
\ddx \left( \frac{5x \tan(x)}{x^2 - 3} \right)
\]
\begin{explanation}
Applying the quotient rule, and the product rule, and the derivative
of cosine:
\begin{align*}
  \ddx &\left( \frac{5x \tan(x)}{x^2 - 3} \right) \\
  &= \frac{(x^2 - 3) \cdot \ddx(\answer[given]{5x \tan(x)}) - 5x \tan(x) \cdot \ddx (\answer[given]{x^2 - 3})}{(x^2 - 3)^2}  \\
  &= \frac{(x^2 - 3)(5 \tan(x) + 5x \answer[given]{\sec^2(x)}) - 5x \tan(x) \cdot 2x}{(x^2 - 3)^2}  \\
  &= \frac{5(x^2-3)(\tan(x)+x \sec^2(x)) - 10x^2 \tan(x)}{(x^2-3)^2}
\end{align*}
\end{explanation}
\end{example}

Finally, we have:

\begin{theorem}[The derivative of secant]\index{derivative!of secant}
\[
\ddx \sec(x) = \sec(x)\tan(x).
\]


\begin{explanation}
We'll rewrite $\sec(x)$ as $(\cos(x))^{-1}$ and use the power rule and the chain rule. Write
\begin{align*}
\ddx \sec(x) &= \ddx(\cos (x))^{-1}\\
&=-1(\cos(x))^{-2}(\answer[given]{-\sin(x)}) \\
&= \frac{\sin(x)}{\cos^2(x)} \\
&= \frac{1}{\cos(x)} \cdot \frac{\sin(x)}{\cos(x)}  \\
&= \sec(x)\tan(x).
\end{align*}
\end{explanation}
\end{theorem}

The derivatives of the cotangent and cosecant are similar and left as
exercises.  Putting this all together, we have:

\begin{theorem}[The Derivatives of Trigonometric Functions] \hfil
\begin{itemize}
\item $\ddx \sin(x) = \cos(x)$.
\item $\ddx \cos(x) = -\sin(x)$.
\item $\ddx \tan(x) = \sec^2(x)$.
\item $\ddx \sec(x) = \sec(x)\tan(x)$.
\item $\ddx \csc(x) = -\csc(x)\cot(x)$.
\item $\ddx \cot(x) = -\csc^2(x)$.
\end{itemize}
\end{theorem}

\begin{example}
Compute:
\[
\eval{\ddx ( \csc(x) \cot(x) )}_{x=\frac{\pi}{3}}
\]
\begin{explanation}
Applying the product rule the facts above, we know that
\[
\ddx ( \csc(x) \cot(x) ) = - \csc^3(x) - \cot^2(x)\answer[given]{\csc(x)}
\]
and so
\[
\eval{\ddx ( \csc(x) \cot(x) )}_{x=\frac{\pi}{3}}
\]
\begin{align*}
  &= \eval{  - \csc^3(x) - \cot^2(x) \answer[given]{\csc(x)}}_{x=\frac{\pi}{3}}  \\
&= - \frac{8}{3 \sqrt{3}} - \frac{1}{3}\cdot \answer[given]{2/\sqrt{3}}
\end{align*}
\end{explanation}
\end{example}





\end{document}
