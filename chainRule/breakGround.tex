\documentclass{ximera}

\input{../preamble.tex}

\outcome{Recognize a composition of functions.}
\outcome{Understand rate of change when quantities are dependent upon each other.}
\outcome{Apply chain rule to relate quantities expressed with different units.}

\title[Break-Ground:]{An unnoticed composition}

\begin{document}
\begin{abstract}
Two young mathematicians discuss the chain rule.
\end{abstract}
\maketitle

Check out this dialogue between two calculus students (based on a true
story):

\begin{dialogue}
\item[Devyn] Riley! Something is bothering me. 
\item[Riley] What is it?
\item[Devyn] I have broken calculus. 
\item[Riley] How?
\item[Devyn] Check this out, we know that:
  \[
  \dd{\theta} \sin(\theta) = \cos(\theta)
  \]
\item[Riley] Right. 
\item[Devyn] But I also know that the derivative \textbf{is} the slope
  of the tangent line at a point.
\item[Riley] Right!
\item[Devyn] But check out this plot of $\sin(\theta)$ that's zoomed in around zero:
  \begin{image}
    \begin{tikzpicture}
      \begin{axis}[
          xmin=-6.75,xmax=6.75,ymin=-1.5,ymax=1.5,
          axis lines=center,
          %ticks=none,
          width=6in,
          height=3in,
          grid style={dashed, gridColor},
          every axis y label/.style={at=(current axis.above origin),anchor=south},
          every axis x label/.style={at=(current axis.right of origin),anchor=west},
        ]        
        \addplot [very thick, penColor, smooth, domain=(-6.75:6.75)] {sin(x)};
      \end{axis}
    \end{tikzpicture}
  \end{image}
\item[Riley] Ok\dots
\item[Devyn] Well, $\cos(0) = 1$, but the slope of this line is
  totally \textbf{not one}! What's going on here?
\end{dialogue}

This problem that Riley and Devyn are having is somewhat subtle. 

\begin{problem}
  What mistake is being made?

\begin{multipleChoice}
\choice{Riley did not plot $\sin(\theta)$.}
\choice{Riley did not take the derivative correctly.}
\choice[correct]{Riley was working in degrees, not radians.}
\choice{$\cos(0)\ne 1$}
\choice{Riley computed the slope incorrectly.}
\end{multipleChoice}
\begin{feedback}
  In calculus, we work with radians. Working in degrees will produce
  erroneous answers.
\end{feedback}
\end{problem}


%\input{../leveledQuestions.tex}


\end{document}
