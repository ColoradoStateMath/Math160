\documentclass{ximera}

\input{../preamble.tex}

\title[Prerequisite Videos: ]{Trigonometry: Solving Equations}

\outcome{Understand how to solve trigonometric equations.}

\begin{document}
\begin{abstract}
  In this series of videos, we will be aimed at learning how
  solving trigonometric equations.
\end{abstract}
\maketitle

The following videos will cover the solving trigonometric equations:

%% Introduction Video
\textbf{Introduction to Trigonometric Equations: What is a trigonometric
equation and how to use the unit circle to solve trigonometric equations.}
%% Labeling this expandable option
\begin{flushright}
{\color{blue}(\emph{Click the arrow to the right to see the Introduction video.})}
\end{flushright}
\begin{center}
\begin{expandable}
\youtube{onK2yNoqYlI}
\end{expandable}
\end{center}


%% Question 1
\textbf{Question 1: Solving  basic trigonometric equations}
\begin{question}
%% Labeling this expandable option
\begin{flushright}
{\color{blue}(\emph{Click the arrow to the right to see the  first question.})}
\end{flushright}
\begin{center}
\begin{expandable}
\youtube{3tL49OoK_TQ}
%% Multiple Choice Question 1
{\color{blue}(\emph{Click the arrow to the right to see the answers 
to the question posed at the end of the video.})}
\begin{expandable}
What are all solutions to the equation $\sec{x} - \sqrt{2} = 0$?
\begin{multipleChoice}
\choice[correct]{$\frac{\pi}{4} + 2\pi n$ and $\frac{7\pi}{4} + 2\pi n$ for all integers $n$}
\choice{$\frac{\pi}{4} + \pi n$ and $\frac{7\pi}{4} + \pi n$ for all integers $n$}
\choice{$\frac{\pi}{4}$ and $\frac{7\pi}{4}$}
\choice{$\frac{4}{\pi} + 2\pi n$ and $\frac{4}{7\pi} + 2\pi n$ for all integers $n$}
\choice{$\frac{4}{\pi} + \pi n$ and $\frac{4}{7\pi} + \pi n$ for all integers $n$}
\choice{$\frac{4}{\pi}$ and $\frac{4}{7\pi}$}
\choice{$\frac{\pi}{4} + 2\pi n$ and $\frac{3\pi}{4} + 2\pi n$ for all integers $n$}
\choice{$\frac{\pi}{4} + \pi n$ and $\frac{3\pi}{4} + \pi n$ for all integers $n$}
\choice{$\frac{\pi}{4}$ and $\frac{3\pi}{4}$}
\choice{$\frac{3\pi}{4} + 2\pi n$ and $\frac{5\pi}{4} + 2\pi n$ for all integers $n$}
\choice{$\frac{3\pi}{4} + \pi n$ and $\frac{5\pi}{4} + \pi n$ for all integers $n$}
\choice{$\frac{3\pi}{4}$ and $\frac{5\pi}{4}$}
\choice{$\frac{5\pi}{4} + 2\pi n$ and $\frac{7\pi}{4} + 2\pi n$ for all integers $n$}
\choice{$\frac{5\pi}{4} + \pi n$ and $\frac{7\pi}{4} + \pi n$ for all integers $n$}
\choice{$\frac{5\pi}{4}$ and $\frac{7\pi}{4}$}
\choice{There is no way to find all of the solutions.}
\choice{None of the above.}
\end{multipleChoice}
%% Example 1
\begin{flushright}
{\color{blue}(\emph{Click the arrow to the right to see an example.})}
\end{flushright}
\begin{expandable}
Example 1
\youtube{nd6pDDs8PqM}
\end{expandable}
\end{expandable}
\end{expandable}
\end{center}
\end{question}



%% Question 2
\textbf{Question 2: Solving trigonometric equations}
\begin{question}
%% Labeling this expandable option
\begin{flushright}
{\color{blue}(\emph{Click the arrow to the right to see the second question.})}
\end{flushright}
\begin{center}
\begin{expandable}
\youtube{H3Qk9VtT1Rw}
%% Multiple Choice Question 2
{\color{blue}(\emph{Click the arrow to the right to see the answers 
to the question posed at the end of the video.})}
\begin{expandable}
What are all solutions to $\sin(\theta) + \cos(\theta)=0$?
\begin{multipleChoice}
\choice[correct]{$\frac{\pi}{4} + 2\pi n$ and $\frac{5\pi}{4} + 2\pi n$ for all integers $n$}
\choice[correct]{$\frac{\pi}{4} + \pi n$ for all integers $n$}
\choice{}
\choice[correct]{$\frac{\pi}{4} + 2\pi n$ and $\frac{5\pi}{4} + 2\pi n$ for all integers $n$}
\choice{$\frac{\pi}{4}$ and $\frac{7\pi}{4}$}
\choice{$\frac{4}{\pi} + 2\pi n$ and $\frac{4}{7\pi} + 2\pi n$ for all integers $n$}
\choice{$\frac{4}{\pi} + \pi n$ and $\frac{4}{7\pi} + \pi n$ for all integers $n$}
\choice{$\frac{4}{\pi}$ and $\frac{4}{7\pi}$}
\choice{$\frac{\pi}{4} + 2\pi n$ and $\frac{3\pi}{4} + 2\pi n$ for all integers $n$}
\choice{$\frac{\pi}{4} + \pi n$ and $\frac{3\pi}{4} + \pi n$ for all integers $n$}
\choice{$\frac{\pi}{4}$ and $\frac{3\pi}{4}$}
\choice{$\frac{3\pi}{4} + 2\pi n$ and $\frac{5\pi}{4} + 2\pi n$ for all integers $n$}
\choice{$\frac{3\pi}{4} + \pi n$ and $\frac{5\pi}{4} + \pi n$ for all integers $n$}
\choice{$\frac{3\pi}{4}$ and $\frac{5\pi}{4}$}
\choice{$\frac{5\pi}{4} + 2\pi n$ and $\frac{7\pi}{4} + 2\pi n$ for all integers $n$}
\choice{$\frac{5\pi}{4} + \pi n$ and $\frac{7\pi}{4} + \pi n$ for all integers $n$}
\choice{$\frac{5\pi}{4}$ and $\frac{7\pi}{4}$}
\choice{There is no way to find all of the solutions.}
\choice{None of the above.}
\end{multipleChoice}
\end{expandable}
\end{expandable}
\end{center}
\end{question}



\end{document}
