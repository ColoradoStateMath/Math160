\documentclass{ximera}

\input{../preamble.tex}

\title[Prerequisite Videos: ]{Introduction to Logarithms}

\outcome{Understand logarithmic functions.}

\begin{document}
\begin{abstract}
  In this series of videos, we will explore how logarithmic functions
  work and what they are.
\end{abstract}
\maketitle

The following videos talk about logarithms:

%% Introduction and Question 1
\textbf{Introduction and Question 1: The definition of logarithm and what
logarithms means}
\begin{question}
%% Labeling this expandable option
\begin{flushright}
{\color{blue}(\emph{Click the arrow to the right to see the Introduction video and first question.})}
\end{flushright}
\begin{center}
\begin{expandable}
\youtube{p9bSDw_Dxqo}
%% Multiple Choice Question 1
{\color{blue}(\emph{Click the arrow to the right to see the answers 
to the question posed at the end of the video.})}
\begin{expandable}
What is $\log_5(125)$?
\begin{prompt}
\[
\log_5(125) \text{ is exactly }\answer[given]{3}
\]
\end{prompt}
%% Example 1
\begin{flushright}
{\color{blue}(\emph{Click the arrow to the right to see an example.})}
\end{flushright}
\begin{expandable}
Example 1
\youtube{3K-sipQyZrQ}
\end{expandable}
\end{expandable}
\end{expandable}
\end{center}
\end{question}


%% Question 2
\textbf{Question 2: Logarithms and the number $1$}
\begin{question}
%% Labeling this expandable option
\begin{flushright}
{\color{blue}(\emph{Click the arrow to the right to see the second question.})}
\end{flushright}
\begin{center}
\begin{expandable}
\youtube{2CEPHloxJ9E}
%% Multiple Choice Question 2
{\color{blue}(\emph{Click the arrow to the right to see the answers 
to the question posed at the end of the video.})}
\begin{expandable}
What is $\log_4(1)$?
\begin{prompt}
\[
\log_4(1) \text{ is exactly }\answer[given]{0}
\]
\end{prompt}
%% Example 2
\begin{flushright}
{\color{blue}(\emph{Click the arrow to the right to see an example.})}
\end{flushright}
\begin{expandable}
Example 2
\youtube{887-Dbnlhmc}
\end{expandable}
\end{expandable}
\end{expandable}
\end{center}
\end{question}


%% Question 3
\textbf{Question 3: Logarithms, fractions, and reciprocals}
\begin{question}
%% Labeling this expandable option
\begin{flushright}
{\color{blue}(\emph{Click the arrow to the right to see the third question.})}
\end{flushright}
\begin{center}
\begin{expandable}
\youtube{lMX5Mg0wNz4}
%% Multiple Choice Question 3
{\color{blue}(\emph{Click the arrow to the right to see the answers 
to the question posed at the end of the video.})}
\begin{expandable}
What is $\log_2\Big(\frac{1}{32}\Big)$?
\begin{prompt}
\[
\log_2\Big(\frac{1}{32} \Big) \text{ is exactly }\answer[given]{-5}
\]
\end{prompt}
%% Example 3
\begin{flushright}
{\color{blue}(\emph{Click the arrow to the right to see an example.})}
\end{flushright}
\begin{expandable}
Example 3
\youtube{ksDJFkogOwo}
\end{expandable}
\end{expandable}
\end{expandable}
\end{center}
\end{question}



%% Properties of Logarithmic
\textbf{Properties of Logarithms: Dealing with exponentials, products,
and division inside your logarithmic function}
\begin{explanation}
%% Labeling this expandable option
\begin{flushright}
{\color{blue}(\emph{Click the arrow to the right to see the nifty properties of logarithms.})}
\end{flushright}
\begin{center}
\begin{expandable}
\youtube{xjutyDI73m0}
{\color{blue}(\emph{Click the arrow to the right to see the property
about dealing with exponents inside of your logarithmic function from the video.})}
\begin{expandable}
Property 1
\[
\log_b(a^m) = m\log_b(a) \text{ when }a,b>0\text{ and }b\neq 1
\]
\end{expandable}
{\color{blue}(\emph{Click the arrow to the right to see the property
about dealing with exponents inside of your logarithmic function from the video.})}
\begin{expandable}
Property 2
\[
\log_b (xy) = \log_b(x) + \log_b(y) \text{ when }b\neq 1, \text{ and both }x \text{ and }y\text{ are constants.}
\]
\end{expandable}
{\color{blue}(\emph{Click the arrow to the right to see the property
about dealing with exponents inside of your logarithmic function from the video.})}
\begin{expandable}
Property 3
\[
\log_b \Big(\frac{x}{y}\Big) = \log_b(x) - \log_b(y) \text{ when }b,x,y>0,\text{ and }b\neq 1.
\]
\end{expandable}
\end{expandable}
\end{center}
\end{explanation}



%% Question 4
\textbf{Question 4: Using properties of logarithms}
\begin{question}
%% Labeling this expandable option
\begin{flushright}
{\color{blue}(\emph{Click the arrow to the right to see the fourth question.})}
\end{flushright}
\begin{center}
\begin{expandable}
\youtube{fM98tz2h3hk}
%% Multiple Choice Question 4
{\color{blue}(\emph{Click the arrow to the right to see the answers 
to the question posed at the end of the video.})}
\begin{expandable}
The number $\log_2(56)$ can be written as a sum of constant numbers
and  logarithms where the inputs of the log base $2$ functions are prime.
Simplify the expression so it is a sum of constants and logarithmic
expressions of the form $n\log_2(p)$ where $p$ is prime. If a portion
of the logarithmic expressions can be simplified to an integer, do so.\\
\begin{center}
$\log_2(56)$ can be written as a sum of $\answer[given]{1}$ term(s)
$n\log_2(p)$ (where the $p$'s are distinct primes) and a constant integer.
\end{center}
\begin{feedback}
\[
\log_2(56) = \answer[given]{1}\cdot \log_2 (\answer[given]{7}) + \answer[given]{3}
\]
\end{feedback}
%% Example 4
\begin{flushright}
{\color{blue}(\emph{Click the arrow to the right to see an example.})}
\end{flushright}
\begin{expandable}
Example 4
\youtube{Sb6SHhwMagM}
\end{expandable}
\end{expandable}
\end{expandable}
\end{center}
\end{question}


\end{document}
