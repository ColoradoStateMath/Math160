\documentclass{ximera}

\input{../preamble.tex}

\outcome{Compute basic antiderivatives.}
\outcome{Solve basic initial value problems.}
\title[Break-Ground:]{Jeopardy! Of calculus}

\begin{document}
\begin{abstract}
  Two young mathematicians discuss a `\textit{Jeopardy!}' version of calculus.
\end{abstract}
\maketitle

Check out this dialogue between two calculus students (based on a true story):

% BADBAD: Jeopardy of Calculus

\begin{dialogue}
\item[Devyn] (Pretending to Alex Trebek) I've got a new costume.
\item[Riley] Whoa! You look just like Trebek!
\item[Devyn] In \textit{Jeopardy!}, I, Trebek, give you an answer, and you must tell me the question.
\item[Riley] Uh Alex, `What are the rules of \textit{Jeopardy!}?'
\item[Devyn] Ha. Exactly! Let's play a different version where I'll
  tell you a derivative, and you tell me the function.  Are you ready?
\item[Riley] I'll take ``Formulas for slope'' for $\$200$.
\item[Devyn] $3\cdot 2x$  
\item[Riley] I've got an answer!  Actually, I've got three different
  answers, I mean questions!
  \begin{enumerate}
  \item ``What's the derivative of $x^2$?
  \item ``What's the derivative of $x^2+1$?
  \item ``What's the derivative of $x^2-1$?
  \end{enumerate}
\item[Devyn] Hmmm. Now I'm not sure which one it was.
%%\item[Riley] What about if you had given me $\frac{\sin(x)}{x}$?
\end{dialogue}


\begin{problem}
  How many functions whose derivative is $x^2$ are there?
  \begin{multipleChoice}
    \choice{Zero}
    \choice{One}
    \choice{Two}
    \choice{Three}
    \choice{Four}
    \choice[correct]{Infinitely many}
  \end{multipleChoice}
\end{problem}

\begin{problem}
  How many functions whose derivative is $x^2$ that equal
  $1$ at $x=0$ are there?
  \begin{multipleChoice}
    \choice{Zero}
    \choice[correct]{One}
    \choice{Two}
    \choice{Three}
    \choice{Four}
    \choice{Infinitely many}
  \end{multipleChoice}
\end{problem}


\input{../leveledQuestions.tex}

\end{document}
