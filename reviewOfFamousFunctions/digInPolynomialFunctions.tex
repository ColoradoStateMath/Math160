\documentclass{ximera}

\input{../preamble.tex}

\outcome{Know the graphs and properties of ``famous'' functions.}

\title[Dig-In:]{Polynomial functions}


\begin{document}
\begin{abstract}
  Polynomials are some of our favorite functions. 
\end{abstract}
\maketitle


The functions you are most familiar with are probably polynomial
functions.

\section{What are polynomial functions?}

\begin{definition}
  A \dfn{polynomial function} in the variable $x$ is a function
  which can be written in the form
  \[
  f(x) = a_nx^n + a_{n-1}x^{n-1} + \dots + a_1 x + a_0
  \]
  where the $a_i$'s are all constants (called the \dfn{coefficients})
  and $n$ is a whole number (called the \dfn{degree} when $n\ne
  0$). The domain of a polynomial function is $(-\infty,\infty)$.
\end{definition}

\begin{question}
  Which of the following are polynomial functions?
  \begin{selectAll}
    \choice[correct]{$f(x) = 0$}
    \choice[correct]{$f(x) = -9$}
    \choice[correct]{$f(x) = 3x+1$}
    \choice{$f(x) = x^{1/2}-x +8$}
    \choice{$f(x) = -4x^{-3}+5x^{-1}+7-18x^2$}
    \choice[correct]{$f(x) = (x+1)(x-1)+x^2-3$}
    \choice{$f(x) = \frac{x^2 - 3x + 2}{x-2}$}
    \choice[correct]{$f(x) = x^7-32x^6-\pi x^3+45/84$}
  \end{selectAll}
\end{question}

The phrase above ``in the variable $x$'' can actually change.
\[
y^2-4y +1
\]
is a polynomial in $y$, and
\[
\sin^2(x) + \sin(x) -3 
\]
is a polynomial in $\sin(x)$.


\section{What can the graphs look like?}

Fun fact:

\begin{theorem}[The Fundamental Theorem of Algebra]\index{Fundamental Theorem of Algebra}
  Every polynomial of the form
  \[
  a_n x^n + a_{n-1} x^{n-1} + \dots + a_1 x + a_0
  \]
  where the $a_i$'s are real (or even complex!) numbers and $a_n \ne 0$ has exactly
  $n$ (possibly repeated) complex roots.
\end{theorem}

Remember, a \dfn{root} is where a polynomial is zero. The theorem
above is a deep fact of mathematics. The great mathematician Gauss
%(spelled Gau\ss\ for fancy people)
proved the theorem in 1799 for his
doctoral thesis. 

The upshot as far as we are concerned is that when we plot a
polynomial of degree $n$, its graph will cross the $x$-axis at most
$n$ times.

\begin{example}
  Here we see the the graphs of four polynomial functions.
  \begin{image}
    \begin{tabular}{cc}
      \begin{tikzpicture}
        \begin{axis}[
          domain=-2:2,
          xmin=-2, xmax=2,
          ymin=-2, ymax=2,
          width=2.5in,
          axis lines =middle, xlabel=$x$, ylabel=$y$,
          every axis y label/.style={at=(current axis.above origin),anchor=south},
          every axis x label/.style={at=(current axis.right of origin),anchor=west},
          ]
	  \addplot [very thick, penColor, smooth] {5*x^5-5*x^4-5*x^3+5*x^2+.5*x -1};
          \node at (axis cs:1.2, 1 ) [penColor,anchor=west] {$A$};
        \end{axis}
      \end{tikzpicture}
      &
      \begin{tikzpicture}
        \begin{axis}[
          domain=-2:2,
          xmin=-2, xmax=2,
          ymin=-2, ymax=2,
          width=2.5in,
          axis lines =middle, xlabel=$x$, ylabel=$y$,
          every axis y label/.style={at=(current axis.above origin),anchor=south},
          every axis x label/.style={at=(current axis.right of origin),anchor=west},
          ]
	  \addplot [very thick, penColor2, smooth] {-5*x^5+5*x^4+5*x^3-4.25*x^2-.3*x +.5};
          \node at (axis cs:1.2, 1 ) [penColor2,anchor=west] {$B$};
        \end{axis}
      \end{tikzpicture}\\
      \begin{tikzpicture}
        \begin{axis}[
          domain=-2:2,
          xmin=-2, xmax=2,
          ymin=-2, ymax=2,
          width=2.5in,
          axis lines =middle, xlabel=$x$, ylabel=$y$,
          every axis y label/.style={at=(current axis.above origin),anchor=south},
          every axis x label/.style={at=(current axis.right of origin),anchor=west},
          ]
	  \addplot [very thick, penColor3, smooth] {5*x^6-5*x^5-5*x^4+5*x^3+x^2 -.5};
          \node at (axis cs:1.2, 1 ) [penColor3,anchor=west] {$C$};
        \end{axis}
      \end{tikzpicture}
      &
      \begin{tikzpicture}
        \begin{axis}[
          domain=-2:2,
          xmin=-2, xmax=2,
          ymin=-2, ymax=2,
          width=2.5in,
          axis lines =middle, xlabel=$x$, ylabel=$y$,
          every axis y label/.style={at=(current axis.above origin),anchor=south},
          every axis x label/.style={at=(current axis.right of origin),anchor=west},
          ]
	  \addplot [very thick, penColor4, smooth,samples=100] {-5*x^6+5*x^5+5*x^4-5*x^3-x^2+1.5*x+1};
          \node at (axis cs:1.2, 1 ) [penColor4,anchor=west] {$D$};
        \end{axis}
      \end{tikzpicture}
    \end{tabular}
  \end{image}
  For each of the curves, determine if the polynomial has
  \textbf{even} or \textbf{odd} degree, and if the leading coefficient
  (the one next to the highest power of $x$) of the polynomial is
  \textbf{positive} or \textbf{negative}.
  \begin{explanation}\hfil
    \begin{itemize}
    \item Curve $A$ is defined by an
      \wordChoice{\choice{even}\choice[correct]{odd}} degree
      polynomial with a \wordChoice{\choice[correct]{positive}\choice{negative}}
      leading term.
    \item Curve $B$ is defined by an
      \wordChoice{\choice{even}\choice[correct]{odd}} degree
      polynomial with a
      \wordChoice{\choice{positive}\choice[correct]{negative}} leading
      term.
    \item Curve $C$ is defined by an
      \wordChoice{\choice[correct]{even}\choice{odd}} degree
      polynomial with a \wordChoice{\choice[correct]{positive}\choice{negative}}
      leading term.
    \item Curve $D$ is defined by an
      \wordChoice{\choice[correct]{even}\choice{odd}} degree
      polynomial with a \wordChoice{\choice{positive}\choice[correct]{negative}}
      leading term.
    \end{itemize}
  \end{explanation}
\end{example}


%% \section{Connections to inverse functions}

%% Polynomials only have inverse functions if they are
%% one-to-one. Pratically this means that polynomials of \textit{even}
%% degree are \textit{never} invertible since they are never
%% one-to-one. Odd degree polynomials may or may not be invertible. This
%% is worth repeating in the language of algebra, and we will write it as
%% a warning.

%% \begin{warning}
%%   Consider $f(x) = x^2$. Suppose we tried to find the inverse of
%%   $x^2$. In this case we plug $f^{-1}(x)$ into $f$ and write
%%   \begin{align*}
%%     f(f^{-1}(x)) &= \left(f^{-1}(x)\right)^2\\
%%     x &= \left(f^{-1}(x)\right)^2
%%   \end{align*}
%% at this point you may be tempted to say ``$f^{-1}(x) = \sqrt{x}$'' but
%% this is not correct. The correct next line is:
%% \[
%% f^{-1}(x) = \pm \sqrt{x},
%% \]
%% but this \textbf{is not a function} from the real numbers to the real
%% numbers.
%% \end{warning}


%% $x^\text{even}$ vs $x^\text{odd}$ and simple nth roots



\end{document}
